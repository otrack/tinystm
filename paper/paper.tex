\documentclass{svproc}
\usepackage{times}
\usepackage{format}
\usepackage{macros}
\usepackage{macros_stm}
\usepackage{paper}
\usepackage{url}
\usepackage[colorlinks=true,linkcolor=blue,urlcolor=blue,citecolor=blue]{hyperref}
\begin{document}

\newboolean{showcomments}
\setboolean{showcomments}{false}
\ifthenelse{\boolean{showcomments}}
{ \newcommand{\mynote}[3]{
   \fbox{\bfseries\sffamily\scriptsize#1}
   {\small$\blacktriangleright$\textsf{\emph{\color{#3}{#2}}}$\blacktriangleleft$}}}
{ \newcommand{\mynote}[3]{}}
% One command per author:
\newcommand{\pf}[1]{\mynote{Pierre}{#1}{red}}
\newcommand{\hm}[1]{\mynote{Patrick}{#1}{pink}}
\newcommand{\vs}[1]{\mynote{Valerio}{#1}{blue}}
\newcommand{\ft}[1]{\mynote{Francois}{#1}{green}}

% tentative results 
% https://docs.google.com/spreadsheets/d/1jqpC7ehu48kJc96EYcEFRbjgV9wq8E9HbdZpwREfw7w/edit#gid=1043149494
% slides from euro-tm 2014
% https://docs.google.com/presentation/d/1AFntDVRBWmjOMjNzX90ZGo0cG4Jd7Pewas4CeemD5TQ/edit#slide=id.p

%\author{(authors omitted due to double-blind review policy)}

\title{Boostring Transactional Memory\\with Stricter Serializability}

\author{Pierre Sutra\inst{2} \and Patrick Marlier\inst{1} \and
 Valerio Schiavoni\inst{1} \and Fran\c{c}ois Trahay\inst{2}}
\institute{
University of Neuch\^atel, Switzerland
\email{\{patrick.marlier,valerio.schiavoni\}@unine.ch}
\and
T\'el\'ecom SudParis, France
\email{\{pierre.sutra,francois.trahay\}@telecom-sudparis.eu}
}

\maketitle

\begin{abstract}
  Software transactional memory (STM) guarantees that a sequence of operations encapsulated into a transaction is atomic.
  This simple yet powerful paradigm is a promising direction for writing concurrent applications.
  %
  Recent STM designs employ a time-based approach to leverage the performance advantage of invisible reads.
  With the advent of many-core architectures and non-uniform memory architecture (NUMA),
  this design is however hitting the synchronization wall of the cache coherency protocol.
  %
  In this paper, we propose a novel and flexible approach to address this limitation.
  Our idea is to leverage the parallelism and the locality of applicative workloads to execute few global operations.
  To that end, we introduce a new STM design that supports invisible reads, lazy snapshots 
  and can be tailored to be either disjoint access parallel, or NUMA-aware.
  Several empirical comparisons against a well-established STM design demonstrate that our design greatly improve perfmance\vs{check it matchs eval content}.
\end{abstract}

\section{Introduction}
\labsection{introduction}

% why TM
The advent of chip level multiprocessing in commodity hardware has pushed applications to be more and more parallel in order to leverage the increase of computational power.
However, the art of concurrent programming is known to be a difficult task~\cite{Lee:2006:PT:1137232.1137289}, and new paradigms are required to help the programmer.
Software Transactional Memory (TM) is widely considered as a promising paradigm in this direction, in particular thanks to its simplicity and programmer's friendliness~\cite{Dragojevic:2011:WSM:1924421.1924440}.

% the case for invisible optimistic reads and DAP
The engine that orchestrates concurrent transactions run by the application, i.e., the concurrency manager, is one of the core aspects of a STM implementation.
A large number of concurrency manager implementations exists, ranging from pessimistic lock-based implementations~\cite{harris2005revocable,afek2012pessimistic} to completely optimistic ones~\cite{hassan2014optimistic}, with~\cite{perelman2011smv} or without multi-version support~\cite{attiya2012single}.
Because application workloads exhibit in general a high degree of parallelism\vs{is this a known fact? for which applications? a ref that assess this statement would be good imho}, these designs tend to favor optimistic concurrency control.
In particular, a widely accepted approach consists in executing tentatively invisible read operations and validating them on the course of the transaction execution to enforce consistency.
Another property of interest is disjoint-access parallelism (DAP) \cite{}.
DAP ensures that concurrent transactions operating on disjoint part of the application do not contend in the concurrency manager.
This property is key to ensures that the system scales with the numbers of cores.

% why OPA
From a developper point of view, the interleaving of transactions must satisfy some form of correctness.
Strict serializability (SSER) is a consistency criteria commonly encountered in database litterature.
This criteria ensures committed transactions behave as if they were executed sequentially, in an order compatible with real-time.
Unfortunately, SSER does not say anything about transaction that abort.
To illustrate this point, let us consider history $h_1$ where transaction $T_1=r(x);r(y)$ and $T_2=w(x);w(y)$ execute concurrently.
In this history $T_1$ aborts after reading inconsistent values for $x$ and $y$, yet this history $h_1$ is allowed by SSER 
\begin{figure}[!h]
  \centering
  \fontsize{8}{11}\selectfont
  \begin{tikzpicture}[scale=0.77]
    \node at (10,.3) {$(h_1)$};
    
    \node at (0.2,1.8) {$T_1$};
    \node at (0.2,1) {$T_2$};
    
    \path[->] (0.5,1) edge (10,1);
    \path[->] (0.5,1.8) edge (10,1.8);
    
    \path[callA] (1.5,1.8) edge (2.5,1.8);
    \path[callA,->] (1.5,2) edge (1.5,1.8);
    \path[callA,->] (2.5,1.8) edge (2.5,2);
    \node at (1.5,2.2) {$r_1(x_0)$}; 
    \node at (2.5,2.2) {};

    \path[callA] (4,1.8) edge (5,1.8);
    \path[callA,->] (4,2) edge (4,1.8);
    \path[callA,->] (5,1.8) edge (5,2);
    \node at (4,2.2) {$r_1(y_2)$};
    \node at (5,2.2) {};

    \path[callA] (7,1.8) edge (8,1.8);
    \path[callA,->] (7,2) edge (7,1.8);
    \path[callA,->] (8,1.8) edge (8,2);
    \node at (7,2.2) {$\flagAbort$};
    \node at (8,2.2) {};

    \path[callB] (2.5,1) edge (3.5,1);
    \path[callB,->] (2.5,1.2) edge (2.5,1);
    \path[callB,->] (3.5,1) edge (3.5,1.2);
    \node at (2.5,1.4) {$w_2(x_2)$};
    \node at (3.5,1.4) {};

    \path[callB] (4.5,1) edge (5.5,1);
    \path[callB,->] (4.5,1.2) edge (4.5,1);
    \path[callB,->] (5.5,1) edge (5.5,1.2);
    \node at (4.5,1.4) {$w_2(y_2)$};
    \node at (5.5,1.4) {};

    \path[callB] (7,1) edge (8,1);
    \path[callB,->] (7,1.2) edge (7,1);
    \path[callB,->] (8,1) edge (8,1.2);
    \node at (7,1.4) {$\flagCommit$};
    \node at (8,1.4) {};
    
    \pgfresetboundingbox
    \clip[use as bounding box] (0,.7) rectangle (10,2);
  \end{tikzpicture}
\end{figure}

Opacity (OPA) was introduced to avoid the side-effects of so-called doom transactions, i.e., transactions which eventually abort (such as $T_1$ in history $h_1$).%
\footnote{  
  Allowing $T_1$ to return both $x_0$ and $y_2$ may have dire consequences in a non-managed environement.
  % def. notion of managed env., that is an environment à la PL/SQL where object are sronngly typed and exception caught.
  For instance \cite{opa}, transaction $T_1$ may compute a division by $0$, leading the program to crash.
}
In addition to SSER, OPA requires that aborted transactions observe a prefix of the committed transactions.
This is the usual consistency criteria for TM.

% the cost of achieving OPA
Achieving OPA is known to be expensive, even for weak progess properties on the transactions \cite{}.
In particular, ensuring that a transaction always observes a consistent snapshot when read are invisible require to either validate the read set after each read operation, or to rely on a global clock.
The former approach results in a quadratic-time validation complexity.
The latter approach is expensive in multi-core/multi-processors architecture, due to a synchronization wall.

% our contributions
In this paper, we address these shortcomings with a new consistency criteria, named stricter serializability (S+SER).
This criteria extends strict serializability while avoiding the inconsistency depicted in history $h_1$.
We present a matching TM algorithm that ensures DAP, invisible reads, and permits transactions to commit as long as they do do contend with conflicting transactions.
We then validate our design by means of a full implementation and several experiments.
Our result shows that when the workloads is strongly parallelism, our algorithm offers performance close to the optimum.

\textbf{Outline.}
The outline of this paper is as follows.
\refsection{criteria} introduces S+SER.
The algorithm and a formal proof of its correctnesss are presented in ~\refsection{stm}
\refsection{evaluation} presents our extensive evaluation against several benchmarks.
We survey related work in \refsection{relatedwork}.
\refsection{conclusion} closes this paper.

\section{A new consistency criteria}
\labsection{criteria}

In what follows, we present the elements of our system model, as commonly found in textbooks (e.g., \cite{}).
Then, we formulate our notion of stricter serializability.

\subsection{System Model}
\labsection{criteria:model}

Software transactional memory (STM) is a recent paradigm that allows multiple processes to access concurently a memory space.
Each process manipulates registers in the shared memory with the help of transactions.
When a process begins a new transactions, it calls function \stmBeginFunction.
Then, it executes a sequence of \stmReadFunction and \stmWriteFunction operations on the registers according to some internal (not modeled) logic.
At the end of the execution, the process calls \stmTryCommitFunction to terminate the transaction, in which case the transaction may either commit ($\flagCommit$), or abort ($\flagAbort$).

A \emph{history} is a sequence of invocations and responses of the tranasctional operations by one or more processes.
As illustrated below, a history is commonly represented with parallel timelines, where each timeline represents the lifetime of a transaction.
\begin{figure}[!h]
  \centering
  \fontsize{8}{11}\selectfont
  \begin{tikzpicture}[scale=0.77]
    \node at (10,.3) {$(h_2)$};

    \node at (0.2,2.6) {$p$};
    \node at (0.2,1.8) {$q$};
    \node at (0.2,1) {$r$};

    \path[->] (0.5,2.6) edge (10,2.6);
    \path[->] (0.5,1.8) edge (10,1.8);
    \path[->] (0.5,1) edge (10,1);
    
    \path[callA] (2.2,2.6) edge (3.5,2.6);
    \path[callA,->] (2.2,2.8) edge (2.2,2.6);
    \path[callA,->] (3.5,2.6) edge (3.5,2.8);
    \node at (2.2,3) {$w_1(x_1)$}; 

    \path[callB] (1,1.8) edge (1.8,1.8);
    \path[callB,->] (1,2) edge (1,1.8);
    \path[callB,->] (1.8,1.8) edge (1.8,2);
    \node at (1,2.2) {$w_2(x_2)$}; 

    \path[callC] (4,1) edge (5,1);
    \path[callC,->] (4,1.2) edge (4,1);
    \path[callC,->] (5,1) edge (5,1.2);
    \node at (4,1.4) {$r_3(x_1)$};

    \path[callC] (6,1) edge (7,1);
    \path[callC,->] (6,1.2) edge (6,1);
    \path[callC,->] (7,1) edge (7,1.2);
    \node at (6,1.4) {$w_3(y_3)$};

    \path[callD] (3,1.8) edge (6.2,1.8);
    \path[callD,->] (3,2) edge (3,1.8);
    \path[callD,->] (6.2,1.8) edge (6.2,2);
    \node at (3,2.2) {$r_4(y_3)$}; 

    \path[callD] (8,1.8) edge (9,1.8);
    \path[callD,->] (8,2) edge (8,1.8);
    \node at (8,2.2) {$r_4(x_{?})$}; 
    
    \pgfresetboundingbox
    \clip[use as bounding box] (0,.7) rectangle (10,2.8);
  \end{tikzpicture}
\end{figure}

A history induces a real-time order between transactions (denoted $\hb_h$).
This order holds between two transactions $T_i$ and $T_j$ when $T_i$ terminates in $h$ before $T_j$ begins.
For instance in history $h_2$, transaction $T_2=w(x)$ precedes transaction $T_3=r(x)$.
When two transactions are noto related with real-time, they are \emph{concurrent}.

A \emph{version} is the state of a register as produced by the write of a transaction.
When transaction $T_i$ writes to some register $x$, an operation denoted $w_i(x_i)$, it creates a new version $x_i$ of $x$.
Versions allow to uniquely identify the state of a register as observed by a read operation, e.g., $r_3(x_1)$ in $h_2$.
When a transaction $T_i$ reads version $x_j$, we shall say that $T_i$ \emph{read-from} transaction $T_1$.

Given some history $h$ and some register $x$, a version order on $x$ for $h$ is a total order relation over the versions of $x$ in $h$.
By extension, a version order for $h$, is the union of all the version orders for all registers (denoted $\ll_h$)
For instance, in history $h_1$ above,  we may consider the version order $(x_1 \ll_{h_1} x_2)$.

A transaction $T_i$ \emph{depends on} a transaction $T_j$, written $T_i \depends T_k$ when $T_i$ reads a version written by $T_j$, or such a relation holds transitively.
A transaction observes a \emph{consistent snapshot} \cite{Chan:1985} when it never misses the effects of some transaction it depends on.
In other words, transaction $T_i$ in a history $h$ observes a consistent snapshot when
\begin{inparaenum}
\item $T_i$ reads version $x_j$,
\item $T_k$ writes version $x_k$, and 
\item $T_i$ depends on $T_{k}$,
\end{inparaenum}
then version $x_k$ is followed by version $x_j$.
Formally, $((r_i(x_j) \in h) \land (w_k(x_k) \in h) \land (T_i \depends T_k)) \implies (x_k \ll_{h,x} x_j)$, where $\ll_{h}$ is some version order defined for history $h$.

\subsection{Contention and bindings}
\labsection{criteria:bindings}

Contention occurs when the steps taken by two transactions interleave, that is if when the two transactions are concurrent.
When the TM is disjoint-access parallel, transactions that do not access the same base objects (i.e., the internals of the TM).
As a conseuqnece, in such a case contention is harmless.

On the other hand, if the two transactions contend on the same object, then they may slow down each other.
With invisible reads, read contention is free as no base objects is written.
Conversely, if the two transactions are writing then it is possible to detect the contention and abort preventively.
In the case where a read-write conflict occurs, a race occurs between the reading and the writing transaction.
In a TM where read are optimistic the arbitrage is simply that the read operations take either place all before the writes, or after.
In the later case, the transaction is bound to read the version produced by the writting transaction.

\begin{definition}[Binding]
  When a transaction $T_i$ reads-from a concurrent transaction $T_j$, we say that $T_i$ is bound to $T_j$ on $x$.
\end{definition}

When a transaction $T_i$ is bound to another transaction $T_j$, to preserve the consistency of its snapshot, $T_i$ must read the updates of $T_j$ as well as its causal dependencies that are intersecting with its read set.
Tracking this causality is difficult for the conention manager since it requires to inspect the readset, or to rely on a global clock.
We observe that it is simpler if the items read previously by $T_i$ are either read also by $T_j$ or by some dependency of $T_j$.
In which case, we will say that the binding is fair.

\begin{definition}[Fair binding]
  Consider that a transaction $T_i$ is bound to transaction $T_j$ on a register $x$.
  We say that this binding is \emph{fair} when, for every register $y$ read before $x$, either $y$ is read by $T_j$ or by some dependency of $T_j$.
\end{definition}

As an example, consider history $h_1$.
In this history, ...

\subsection{Stricter serializability}
\labsection{criteria:spser}

Stricter serializability (S+SER) is a stronger consistency criteria than strict serializability (SSER).
As SSER, this criteria requires that committed transactions form a sequential history.
In addiition, S+SER prohibits transactions to view inconsistencies unless one of their bindings is unfair.
Below, we give a graph characterization of stricter serializability.

\begin{definition}[Serialization graph]
  Consider some version order $\ll_h$ for $h$.
  We note $<$ the relations over the commmitted transactions in $h$ induced by $\ll_h$; namely:
  \begin{displaymath}
    \begin{array}{l}
      T_i < T_{j \neq i}  \equaldef \\
      \lor~ T_i \hb_h T_j {\hspace{16.8em}\text{(1)}} \\
      \lor~ \exists x : \lor~ r_j(x_i) \in h {\hspace{13.3em}\text{(2)}} \\
      \hspace{2.9em}\lor~ \exists T_k : x_k \ll_{h,x} x_j \land \lor~T_k = T_i {\hspace{5em}\text{(3)}} \\
      \hspace{12.2em} \lor~ r_i(x_k) \in h {\hspace{3.9em}\text{(4)}}
    \end{array}
  \end{displaymath}
\end{definition}

In the above definition, (1) is a real-time order between $T$ and $T'$, (2) a read-write dependency, (3) a version ordering, and (4) an anti-dependency.

\begin{definition}[Stricter serializability]
  A history $h$ belongs to stricter serializability if its serialization graph is acyclic and if some transaction $T_i$ observes an inconsistent snapshot in $h$, then oene of its bindings is unfair.
\end{definition}


\section{Algorithm}
\labsection{stm}

In this section, we present a software transactional memory to implement $\SPSER$.
This construction is disjoint-access parallel and weakly-progressive, i.e., it aborts a transaction only if it encounters a concurrent conflicting transaction 
This section gives an overview of the algorithm, present its internals and justify some design choices.
A correctness proofs follows.

\subsection{Overview}
\labsection{stm:overview}

\refalg{stm} depicts the pseudo-code of our construction of the STM interface at some process $p$.
Our design follow the general design of the lazy snapshot algorithm (LSA) of \citet{FelberFMR10}, replacing the central clock with a more flexible mechanism.
\refalg{stm} employs a deferred update schema that consists in two steps.
A transaction first executes optimistically, buffering its updates.
Then at commit time the transaction is certified and, provided it commits, its updates are applied to the shared memory.

During the execution of a transaction, process $p$ checks that the objects accessed so far did not change.
Similarly to LSA, this check is lazily executed.
\refalg{stm} executes it only when a shared object appears to have been recently updated, or when the transaction terminates.

\subsection{Tracking Time}
\labsection{stm:time}

\refalg{stm} tracks time to compute how concurrent transactions interleave during an execution.
To this end, the algorithm makes use of logical clocks.
We model the interface of a \emph{logical clock} with two operations: $\cread$ returns a value in $\naturalSet$, and $\cadv(v \in \naturalSet)$ updates the clock with value $v$.
The sequential specification of a logical clock guarantees a single property, that the time flows forward:
\begin{inparaenum}
\item[\emph{(Time monoticity)}]
  A read operation always returns at least the greatest value to which the clock advanced so far.
  Formally, for every history $h$, $(\responseAny{\cread}{v} \in h) \implies (v \geq \max{(\{u : \cadv(u) \hb_h \cread \} \union \{0\})})$.
\end{inparaenum}

\refalg{stm} associates logical clocks with both processes and transactions.
To retrieve the clock associated with some object $i$, the algorithm uses function $\clockOf{i}$.
Notice that in the pseudo-code, when it is clear from the context, we write $\clockOf{i}$ as a shorthand for $\clockOf{i}.\mathit{read}()$.

The clock associated with a transaction is always local (\refline{stm:var:1}).
In the case of a process, it might be shared or not (\refline{stm:var:2}).
The flexibility of our design comes from this locality choice for \clockOf{p}.
When the clock is shared, it is linearizable.
To implement an (obstruction-free) linearizable clock we employ the following common approach:
\begin{construction}
  Let $x$ be a shared register initialized to $0$.
  When $\cread$ is called, we return the value stored in $x$.
  Upon executing $\cadv(v)$, we fetch the value stored in $x$, say $u$.
  If $v > u$ holds, we execute a compare-and-swap to replace $u$ with $v$; 
  otherwise the operation returns.
  If the compare-and-swap fails, the previous steps are retried.
\end{construction}

\section{Algorithm}
\labsection{stm}

In this section, we present a software transactional memory to implement $\SPSER$.
This construction is disjoint-access parallel and weakly-progressive, i.e., it aborts a transaction only if it encounters a concurrent conflicting transaction 
This section gives an overview of the algorithm, present its internals and justify some design choices.
A correctness proofs follows.

\subsection{Overview}
\labsection{stm:overview}

\refalg{stm} depicts the pseudo-code of our construction of the STM interface at some process $p$.
Our design follow the general design of the lazy snapshot algorithm (LSA) of \citet{FelberFMR10}, replacing the central clock with a more flexible mechanism.
\refalg{stm} employs a deferred update schema that consists in two steps.
A transaction first executes optimistically, buffering its updates.
Then at commit time the transaction is certified and, provided it commits, its updates are applied to the shared memory.

During the execution of a transaction, process $p$ checks that the objects accessed so far did not change.
Similarly to LSA, this check is lazily executed.
\refalg{stm} executes it only when a shared object appears to have been recently updated, or when the transaction terminates.

\subsection{Tracking Time}
\labsection{stm:time}

\refalg{stm} tracks time to compute how concurrent transactions interleave during an execution.
To this end, the algorithm makes use of logical clocks.
We model the interface of a \emph{logical clock} with two operations: $\cread$ returns a value in $\naturalSet$, and $\cadv(v \in \naturalSet)$ updates the clock with value $v$.
The sequential specification of a logical clock guarantees a single property, that the time flows forward:
\begin{inparaenum}
\item[\emph{(Time monoticity)}]
  A read operation always returns at least the greatest value to which the clock advanced so far.
  Formally, for every history $h$, $(\responseAny{\cread}{v} \in h) \implies (v \geq \max{(\{u : \cadv(u) \hb_h \cread \} \union \{0\})})$.
\end{inparaenum}

\refalg{stm} associates logical clocks with both processes and transactions.
To retrieve the clock associated with some object $i$, the algorithm uses function $\clockOf{i}$.
Notice that in the pseudo-code, when it is clear from the context, we write $\clockOf{i}$ as a shorthand for $\clockOf{i}.\mathit{read}()$.

The clock associated with a transaction is always local (\refline{stm:var:1}).
In the case of a process, it might be shared or not (\refline{stm:var:2}).
The flexibility of our design comes from this locality choice for \clockOf{p}.
When the clock is shared, it is linearizable.
To implement an (obstruction-free) linearizable clock we employ the following common approach:
\begin{construction}
  Let $x$ be a shared register initialized to $0$.
  When $\cread$ is called, we return the value stored in $x$.
  Upon executing $\cadv(v)$, we fetch the value stored in $x$, say $u$.
  If $v > u$ holds, we execute a compare-and-swap to replace $u$ with $v$; 
  otherwise the operation returns.
  If the compare-and-swap fails, the previous steps are retried.
\end{construction}

\section{Algorithm}
\labsection{stm}

In this section, we present a software transactional memory to implement $\SPSER$.
This construction is disjoint-access parallel and weakly-progressive, i.e., it aborts a transaction only if it encounters a concurrent conflicting transaction 
This section gives an overview of the algorithm, present its internals and justify some design choices.
A correctness proofs follows.

\subsection{Overview}
\labsection{stm:overview}

\refalg{stm} depicts the pseudo-code of our construction of the STM interface at some process $p$.
Our design follow the general design of the lazy snapshot algorithm (LSA) of \citet{FelberFMR10}, replacing the central clock with a more flexible mechanism.
\refalg{stm} employs a deferred update schema that consists in two steps.
A transaction first executes optimistically, buffering its updates.
Then at commit time the transaction is certified and, provided it commits, its updates are applied to the shared memory.

During the execution of a transaction, process $p$ checks that the objects accessed so far did not change.
Similarly to LSA, this check is lazily executed.
\refalg{stm} executes it only when a shared object appears to have been recently updated, or when the transaction terminates.

\subsection{Tracking Time}
\labsection{stm:time}

\refalg{stm} tracks time to compute how concurrent transactions interleave during an execution.
To this end, the algorithm makes use of logical clocks.
We model the interface of a \emph{logical clock} with two operations: $\cread$ returns a value in $\naturalSet$, and $\cadv(v \in \naturalSet)$ updates the clock with value $v$.
The sequential specification of a logical clock guarantees a single property, that the time flows forward:
\begin{inparaenum}
\item[\emph{(Time monoticity)}]
  A read operation always returns at least the greatest value to which the clock advanced so far.
  Formally, for every history $h$, $(\responseAny{\cread}{v} \in h) \implies (v \geq \max{(\{u : \cadv(u) \hb_h \cread \} \union \{0\})})$.
\end{inparaenum}

\refalg{stm} associates logical clocks with both processes and transactions.
To retrieve the clock associated with some object $i$, the algorithm uses function $\clockOf{i}$.
Notice that in the pseudo-code, when it is clear from the context, we write $\clockOf{i}$ as a shorthand for $\clockOf{i}.\mathit{read}()$.

The clock associated with a transaction is always local (\refline{stm:var:1}).
In the case of a process, it might be shared or not (\refline{stm:var:2}).
The flexibility of our design comes from this locality choice for \clockOf{p}.
When the clock is shared, it is linearizable.
To implement an (obstruction-free) linearizable clock we employ the following common approach:
\begin{construction}
  Let $x$ be a shared register initialized to $0$.
  When $\cread$ is called, we return the value stored in $x$.
  Upon executing $\cadv(v)$, we fetch the value stored in $x$, say $u$.
  If $v > u$ holds, we execute a compare-and-swap to replace $u$ with $v$; 
  otherwise the operation returns.
  If the compare-and-swap fails, the previous steps are retried.
\end{construction}

\input{algorithms/stm.tex}

\subsection{Details}
\labsection{stm:details}

In \refalg{stm}, each object $x$ has a \emph{location} in the shared memory, denoted $\locationOf{x}$.
This location stores a pair $(t,d)$, where $t \in \naturalSet$ is a \emph{timestamp}, and $d$ is the actual content of $x$ as seen by transactions.
For simplicity, we shall name hereafter a pair $(t,d)$ a \emph{version} of object $x$.
Since the location of object $x$ is unique, a single version of object $x$ may exist at a time in the memory.
As usual, we asume some transaction $\transInit$ that intializes for every object $x$ the location $\locationOf{x}$ to $(0,\bot)$.
Furthermore, we consider that each object $x$ is atomic.

\refalg{stm} associates each object with a lock.
To manipulate the lock-related functions of object $x$, 
a process $p$ employs appropriately the functions $\lock{x}$, $\isLocked{x}$ and $\unlock{x}$.

For every transaction $T$ submitted to the system, \refalg{stm} maintains three local data structures:
\begin{inparaenum}[]
\item $\clockOf{T}$ is the logical clock of transaction $T$,
\item $\readSetOf{T}$ is a map that contains its \emph{read set}, and 
\item $\writeSetOf{T}$ is another map that stores the \emph{write set} of $T$.
\end{inparaenum}
\refalg{stm} updates incrementally $\readSetOf{T}$ and $\writeSetOf{T}$ over the course of the execution.
The read set serves to check that the view of the shared memory, or \emph{snpashot}, seen by the transaction is consistent.
The write set buffers updates.
In detail, the execution of a transaction $T$ proceeds as follows.

\begin{itemize}
\item[-] %
  When $T$ starts its execution, \refalg{stm} initializes $\clockOf{T}$ to the value of $\clockOf{p}$, then both $\readSetOf{T}$ and $\writeSetOf{T}$ to $\emptySet$ (\reflines{stm:start:1}{stm:start:3}).
\item[-] %
  When $T$ accesses a shared object $x$, if $x$ was previously written, its value is returned (\refline{stm:read:1}).
  Otherwise, \refalg{stm} fetches atomically the version $(d,t)$, as seen in location $\locationOf{x}$.
  Then, the algorithm checks that 
  \begin{inparaenum}
  \item no lock is held on $x$, and 
  \item in case $x$ was previously accessed, that $T$ observes the same version.
  \end{inparaenum}
  If one of these two conditions fails, \refalg{stm} aborts transaction $T$ (\refline{stm:read:5}).
  The algorithm then checks that the timestamp $t$ associated to the content $d$ is smaller than the clock of $T$.
  In case this does not hold (\refline{stm:read:6}), \refalg{stm} tries extending the snapshot of $T$ by calling function $\stmExtend{}$.
  This function returns $\true$ when the versions previously read by $T$ are still valid.
  In which case, $\clockOf{T}$ is updated to the value $t$.
  If \refalg{stm} succeeds in extending (if needed) the snapshot of $T$, $d$ is returned and the read set of $T$ updated accordingly;
  otherwise transaction $T$ is aborted (\refline{stm:read:7}).
\item[-] %
  Upon executing a write request on behalf of $T$ to some object $x$, \refalg{stm} takes the lock associated with $x$ (\refline{stm:write:1}), and in case of success, it buffers the update value $d$ in $\writeSetOf{T}$ (\refline{stm:write:6}).
  The timestamp $t$ of $x$ at the time \refalg{stm} takes the lock serves two purposes.
  First, \refalg{stm} checks that $t$ is lower than the current clock of $T$, and if not $T$ is extended (\refline{stm:write:4}).
  Second, it is saved in \writeSetOf{T} to ensure that at commit time the timestamp of the version of $x$ written by $T$ is greater than $t$.
\item[-] %
  When $T$ requests to commit, \refalg{stm} certifies the read set by calling function $\stmExtend{}$ with the clock of $T$ (\refline{stm:try:1}).
  If this test succeeds, transaction $T$ commits (\reflines{stm:commit:1}{stm:commit:6}).
  In such a case, \clockOf{T} ticks to reach its final value (\refline{stm:commit:1}).
  By construction, this value is greater than the clock of process $p$ at the time transaction $T$ started (\refline{stm:start:1}), as well as all the timestamps of all the versions read or written by $T$ (\reflinestwo{stm:read:5}{stm:write:4}).
  \refalg{stm} updates the clock of $p$ with the final value of \clockOf{T} (\refline{stm:commit:2}), then it updates the items written by $T$ with their novel versions (\refline{stm:commit:4}).
\end{itemize}

\refalg{stm} replaces the global clock usually employed in STM architectures with a locality-aware clock.
When \clockOf{p} is local to each process, \refalg{stm} ensures strict disjoint access parallelism (DAP) \cite{Attiya2015}.
More formally, this means that two transaction access concurrently a bae object of the implementation only if they contend on some shared object.
Provided the workload is parallel, DAP ensures the scalability of \refalg{stm}.
We assess empirically this claim in \refsection{evaluation}.

On the other hand, if processes need to synchronize too often, maintaining consistency among the various clocks is expensive.
In this situation, it might be of interest to find a compromise between the cost of cache coherency and the need for synchronization.
For instance, in a NUMA architecture, \refalg{stm} may assign a common clock per hardware socket.
Upon a call to $\clockOf{p}$, the algortihm returns the clock defined for the socket in which the processor executing process $p$ resides.

\subsection{Guarantees}
\labsection{stm:guarantees}

In this section, we prove the different properties \refalg{stm} provides.
First, we show that our STM design is weakly progressive and stricter serializable.
Then, we prove that, when $\clockOf{p}$ is local, \refalg{stm} is disjoint-access parallel.

\paragraph{(Weak-progress)}
A transaction executes under \emph{weak progressiveness} \cite{Guerraoui:2009}, or equivalently it is \emph{weakly progressive}, when it aborts only if it encounters a conflicting transaction.
By extension, an STM is weakly progressive when it only produces histories during which transactions are weakly-progressive.
We prove that this property holds for \refalg{stm}.

In \refalg{stm}, a transaction $T$ aborts either at \refline{stm:read:5}, \ref{line:alg:stm:read:7}, \ref{line:alg:stm:write:2}, \ref{line:alg:stm:write:5}, or \ref{line:alg:stm:try:2}.
We observe that in such a case either $T$ observes an item $x$ locked, or that the timestamp associated with $x$ has changed.
It follows that if $T$ aborts then it observes a conflict with a concurrent transaction.
From which we deduce that it is executing under weak progressiveness.

\paragraph{(Stricter serializability)}
Consider some run $\run$ of \refalg{stm}, and let $h$ be the history produced in \run.
At the light of its pseudo-code, every function defined in \refalg{stm} is wait-free.
As a consequence, we may consider without lack of generality that $h$ is complete, i.e., every transaction executed in $h$ terminates with either a commit or an abort event.
In what follows, we define that $\ll_h$ as the order in which writes to the object locations are linearized in $\rho$.
We first prove that $<$ is acyclic for this definition of $\ll_h$.
Then, we show that, if a transaction does not exhibit any unfair binding, then it observes a strictly consistent snapshot.
For some transaction, we shall note $\clockOf{T_i}_f$ the final value of $\clockOf{T}$.

\begin{proposition}
  \labprop{stm:1}
  Consider two transactions $T_i$ and $T_{j \neq i}$ in $h$.
  If either $T_i \depends T_j$ or  $x_j \ll_h x_i$ holds, then $\clockOf{T_i}_f \geq \clockOf{T_j}_f$ is true.
  In addition, if transaction $T_i$ commits then the ordering is strict, i.e., $\clockOf{T_i}_f > \clockOf{T_j}_f$.
\end{proposition}

\begin{proof}

  In each of the two cases, we prove that $\clockOf{T_i}_f \geq \clockOf{T_i}_f$ holds before transaction $T_i$ commits.
  \begin{itemize}
  \item[($T_i \depends T_j$)]
    Let $x$ be an object such that $r_i(x_j)$ occurs in $h$.
    Since transaction $T_i$ reads version $x_j$, transaction $T_j$ commits.
    We observe that $T_j$ writes version $x_j$ together with $\clockOf{T_j}_f$ at $\locationOf{x}$ when it commits (\refline{stm:commit:4}).
    As a consequence, when transaction $T_i$ returns version $x_i$ at \refline{stm:read:9}, it assigns $\clockOf{T_j}_f$ to $t$ before at \refline{stm:read:2}.
    The condition at \refline{stm:read:6} implies that either $\clockOf{T_i} \geq t$ holds, or a call to $\stmExtend{T_i,t}$ occurs.
    In the latter case, transactino $T_i$ executes \refline{stm:extend:5}, advancing its clock up to the value of $t$.
  \item[($x_j \ll_h x_i$)]
    By definition, relation $\ll_h$ forms a total order over all versions of $x$.
    Thus, we may reason by induction, considering that $x_i$ is immediately after $x_j$ in the order $\ll_h$.
    When $T_j$ returns from $w_j(x_j)$ at \refline{stm:write:6}, it holds a lock on $x$.
    This lock is released at \refline{stm:commit:5} after writing to $\locationOf{x}$.
    As $\ll_h$ follows the linearization order, $T_i$ executes \refline{stm:write:1} after $T_j$ wrote $(x_j, \clockOf{T_j}_f)$ to $\locationOf{x}$.
    Location $\locationOf{x}$ is not updated between $x_j$ and $x_i$.
    Hence, after$T_i$ executes \refline{stm:write:4}, $\clockOf{T_i} \geq \clockOf{T_j}$ holds.
  \end{itemize}
  Since a clock is monotonic, the relation holds forever.
  Then, if transaction $T_i$ commits, it must executes \refline{stm:commit:1}, leading to $\clockOf{T_i}_f > \clockOf{T_i}_f$.
\end{proof}

\begin{proposition}
  \labprop{stm:2}
  Consider two transactions $T_i$ and $T_{j \neq i}$ in $\committed{h}$ such that $T_i < T_j$.
  Then, transaction $T_i$ invokes $\stmCommitFunction$ before transaction $T_j$ in $h$.
\end{proposition}

\begin{proof}
  Assume that $T_i$ and $T_j$ conflict of some object $x$.
  We examine in order each of the four cases defining relation $<$.
  \begin{compactitem}
  \item ($T_i \hb_h T_j$)\\
    This case is immediate.
  \item ($\exists x : r_j(x_i) \in h$)\\
    Before committing, $T_j$ invokes \stmExtendFunction at \refline{stm:try:1}.
    Since $T_j$ commits in $h$, it should retrieve $(x_i,\msgAny)$ from $\locationOf{x}$ when executing \refline{stm:extend:2}.
    Hence, transaction $T_i$ has already exeecuted \refline{stm:commit:4} on object $x$.
    It follows that $T_i$ invokes $\stmCommitFunction$ before transaction $T_j$ in history $h$.
  \item ($\exists x : x_i \ll_h x_j$)\\
    By definition of $\ll_h$, the write of version $x_i$ is linearized before the write of version $x_j$ in $\rho$.
    After $T_i$ returns from $w_i(x_i)$, it owns a lock on object $x$ (\refline{stm:commit:4}).
    The object is then unlocked by transaction $T_i$ at \refline{stm:commit:5}.
    As a consequence, transaction $T_i$ takes a lock on object $x$ after $T_i$ invokes operation $\stmCommitFunction$.
    From which it follows that the claim holds.
  \item ($\exists x, T_k : x_k \ll_x x_j \land r_i(x_k)$)\\
    For the sake of contradiction, assume that $T_j$ invokes $\stmCommitFunction$ before $T_i$.
    When $T_j$ invokes $\stmCommitFunction$, it holds a lock on $x$.
    This lock is released at \refline{stm:commit:5} after version $x_j$ is written at location $\locationOf{x}$.
    %
    Then, consider the time at which $T_i$ invokes $\stmTryCommitFunction$.
    The call at \refline{stm:try:1} leads to fetching $\locationOf{x}$ at \refline{stm:extend:2}.
    Since $T_i$ reads version $x_k$ in $h$, a pair $(\clockOf{T_k}_f, x_k)$ is in $\readSetOf{T_i}$.
    From the definition of $\ll_h$ the write of $(\clockOf{T_k}_f, x_k)$ takes place before the write of version $(\clockOf{T_j}_f,x_j)$ in $\rho$.
    Hence, $\locationOf{x}$ does not contain anymore $(\clockOf{T_k}_f, x_k)$
    Applying \refprop{stm:1}, $T_i$ executes \refline{stm:extend:4} and aborts at \refline{stm:try:3}.
    Contradiction.
  \end{compactitem}
  
\end{proof}

\begin{proposition}
  \labprop{stm:3}
  History $h$ does not exhibit any RC-anti-dependencies ($h \notin \RCAD$)
\end{proposition}

\begin{proof}

  Consider $T_i$, $T_j$ and $T_k$ such that $r_i(x_k), w_j(x_j) \in h$, $x_k \ll_h x_j$ and $T_j$ commits before $T_i$.
  %
  When $T_i$ invokes $\stmTryCommit$, transaction $T_j$ is committed.
  Thus, when $T_i$ executes \refline{stm:try:1} to call $\stmExtendFunction$, it fetches $(x_j,\clockOf{T_j})$ from $\locationOf{x}$.
  On the other hand, $(x_k, \clockOf{T_k}_f) \in \readSetOf{T_i}$ holds at that time.
  From \refprop{stm:1}, $\clockOf{T_j}_f \neq \clockOf{T_k}_f$.
  It follows that the test at \refline{stm:extend:3} fails, leading $T_j$ to abort at \refline{stm:try:2}.  
  
\end{proof}

\begin{theorem}
  \labtheo{spser}
  History $h$ belongs to $\SPSER$.
\end{theorem}

\begin{proof}
  \refprop{stm:2} tells us that if $T_i < T_j$ holds then $T_i$ commits before $T_j$.
  It follows that the strict serialization graph $(\committed{h},<)$ is acyclic.

  Let us now turn our attention to the second property of $\SPSER$.
  To this end, assume that a transaction $T_i$ aborts in $h$.
  For the sake of contradiction, consider that $T_i$ exhibits fair bindings and yet observes a non-strictly consistent snapshot.

  Applying the definition given in \refsection{criteria:model}, there exist transactions $T_j$ and $T_{k \neq j}$ such that $T_i \depends T_j$, $r_i(x_k)$ occurs in $h$ and $x_k \ll_h x_j$.  
  If $T_j \ll_{h} T_i$, then transaction $T_i$ cannot observe version $x_k$.
  Consequently, transaction $T_j$ is concurrent to $T_i$.
  In addition, there exists a transaction $T_l$ and some object $y$ such that $T_i$ performs $r_i(y_l)$ and $T_l \depends T_j$.

  Relation $<$ is acyclic, thus $x_k \neq y_l$ holds.  
  It remains to consider the following two cases:
  \begin{compactitem}
  \item ($r_i(y_l) \hb_h r_i(x_k)$)\\
    From \refprop{stm:2} and $T_l \depends T_j$, transaction $T_j$ is committed at the time $T_i$ reads object $x$.
    Contradiction.
  \item ($r_i(x_k) \hb_h r_i(y_l)$)\\
    We first argue that the timestamp fetches from $\locationOf{y}$ at the time $T_i$ executes \refline{stm:read:2} is greater than $\clockOf{T_i}$.
    \begin{proof}
      First of all, observe that $T_j$ is not committed at the time $T_i$ reads object $x$ (since $x_{k} \ll_h x_{j}$ holds).
      Denoting $q$ the process that executes $T_j$, $\clockOf{q} < \clockOf{T_j}_f$ is true when $T_i$ begins its execution at \refline{stm:start:1}.
      %%    
      Since transaction $T_l$ is concurrent to $T_i$ and $r_i(y_l)$ occurs, $T_i$ is bound to $T_l$ on $y$.
      Assume some object $z$ read by $T_i$ before $y$.
      Because the binding of $T_i$ to $T_l$ is fair, $T_l$ (or one of its dependencies) accesses $z$.
      As $h \in \RCAD$, if some transaction $T_r$ updates to version $z_r$ the object $z$ read by $T_i$ this transaction cannot be concurrent to $T_l$.
      Hence, applying \refprop{stm:1}, $\clockOf{T_r}_f < \clockOf{T_l}_f$ holds.
      This leas to the fact that $\clockOf{T_i} < \clockOf{T_l}_f$ at the time $T_i$ invokes \stmReadFunction on $x$.
      k
    \end{proof}
    Consequently, transaction $T_i$ invokes \stmExtendFunction at \refline{stm:refline:6}.
    Transaction $T_j$ is committed at that time.
    Since $T_j \depends T_l$, $T_j$ is also commited.
    Following a reasoning similar to the one given in \refprop{stm:3}, the test at \refline{stm:extend:3} fails.
  \end{compactitem}
\end{proof}

\paragraph{(Disjoint-access parallelism)}
The logical clocks used in \refalg{stm} can be shared or local to each process.
When they are are local, function $\clockOf{p}$ becomes injective.
Consider such a scenario and two transactions $T_i$ and $T_j$ that do not access on a common object.
If $T_i$ and $T_j$ are executed by distinct processes, then they do not contend on some base object in \refalg{stm}.
It follows that \refalg{stm} is strictly disjoint-access parallel.



\subsection{Details}
\labsection{stm:details}

In \refalg{stm}, each object $x$ has a \emph{location} in the shared memory, denoted $\locationOf{x}$.
This location stores a pair $(t,d)$, where $t \in \naturalSet$ is a \emph{timestamp}, and $d$ is the actual content of $x$ as seen by transactions.
For simplicity, we shall name hereafter a pair $(t,d)$ a \emph{version} of object $x$.
Since the location of object $x$ is unique, a single version of object $x$ may exist at a time in the memory.
As usual, we asume some transaction $\transInit$ that intializes for every object $x$ the location $\locationOf{x}$ to $(0,\bot)$.
Furthermore, we consider that each object $x$ is atomic.

\refalg{stm} associates each object with a lock.
To manipulate the lock-related functions of object $x$, 
a process $p$ employs appropriately the functions $\lock{x}$, $\isLocked{x}$ and $\unlock{x}$.

For every transaction $T$ submitted to the system, \refalg{stm} maintains three local data structures:
\begin{inparaenum}[]
\item $\clockOf{T}$ is the logical clock of transaction $T$,
\item $\readSetOf{T}$ is a map that contains its \emph{read set}, and 
\item $\writeSetOf{T}$ is another map that stores the \emph{write set} of $T$.
\end{inparaenum}
\refalg{stm} updates incrementally $\readSetOf{T}$ and $\writeSetOf{T}$ over the course of the execution.
The read set serves to check that the view of the shared memory, or \emph{snpashot}, seen by the transaction is consistent.
The write set buffers updates.
In detail, the execution of a transaction $T$ proceeds as follows.

\begin{itemize}
\item[-] %
  When $T$ starts its execution, \refalg{stm} initializes $\clockOf{T}$ to the value of $\clockOf{p}$, then both $\readSetOf{T}$ and $\writeSetOf{T}$ to $\emptySet$ (\reflines{stm:start:1}{stm:start:3}).
\item[-] %
  When $T$ accesses a shared object $x$, if $x$ was previously written, its value is returned (\refline{stm:read:1}).
  Otherwise, \refalg{stm} fetches atomically the version $(d,t)$, as seen in location $\locationOf{x}$.
  Then, the algorithm checks that 
  \begin{inparaenum}
  \item no lock is held on $x$, and 
  \item in case $x$ was previously accessed, that $T$ observes the same version.
  \end{inparaenum}
  If one of these two conditions fails, \refalg{stm} aborts transaction $T$ (\refline{stm:read:5}).
  The algorithm then checks that the timestamp $t$ associated to the content $d$ is smaller than the clock of $T$.
  In case this does not hold (\refline{stm:read:6}), \refalg{stm} tries extending the snapshot of $T$ by calling function $\stmExtend{}$.
  This function returns $\true$ when the versions previously read by $T$ are still valid.
  In which case, $\clockOf{T}$ is updated to the value $t$.
  If \refalg{stm} succeeds in extending (if needed) the snapshot of $T$, $d$ is returned and the read set of $T$ updated accordingly;
  otherwise transaction $T$ is aborted (\refline{stm:read:7}).
\item[-] %
  Upon executing a write request on behalf of $T$ to some object $x$, \refalg{stm} takes the lock associated with $x$ (\refline{stm:write:1}), and in case of success, it buffers the update value $d$ in $\writeSetOf{T}$ (\refline{stm:write:6}).
  The timestamp $t$ of $x$ at the time \refalg{stm} takes the lock serves two purposes.
  First, \refalg{stm} checks that $t$ is lower than the current clock of $T$, and if not $T$ is extended (\refline{stm:write:4}).
  Second, it is saved in \writeSetOf{T} to ensure that at commit time the timestamp of the version of $x$ written by $T$ is greater than $t$.
\item[-] %
  When $T$ requests to commit, \refalg{stm} certifies the read set by calling function $\stmExtend{}$ with the clock of $T$ (\refline{stm:try:1}).
  If this test succeeds, transaction $T$ commits (\reflines{stm:commit:1}{stm:commit:6}).
  In such a case, \clockOf{T} ticks to reach its final value (\refline{stm:commit:1}).
  By construction, this value is greater than the clock of process $p$ at the time transaction $T$ started (\refline{stm:start:1}), as well as all the timestamps of all the versions read or written by $T$ (\reflinestwo{stm:read:5}{stm:write:4}).
  \refalg{stm} updates the clock of $p$ with the final value of \clockOf{T} (\refline{stm:commit:2}), then it updates the items written by $T$ with their novel versions (\refline{stm:commit:4}).
\end{itemize}

\refalg{stm} replaces the global clock usually employed in STM architectures with a locality-aware clock.
When \clockOf{p} is local to each process, \refalg{stm} ensures strict disjoint access parallelism (DAP) \cite{Attiya2015}.
More formally, this means that two transaction access concurrently a bae object of the implementation only if they contend on some shared object.
Provided the workload is parallel, DAP ensures the scalability of \refalg{stm}.
We assess empirically this claim in \refsection{evaluation}.

On the other hand, if processes need to synchronize too often, maintaining consistency among the various clocks is expensive.
In this situation, it might be of interest to find a compromise between the cost of cache coherency and the need for synchronization.
For instance, in a NUMA architecture, \refalg{stm} may assign a common clock per hardware socket.
Upon a call to $\clockOf{p}$, the algortihm returns the clock defined for the socket in which the processor executing process $p$ resides.

\subsection{Guarantees}
\labsection{stm:guarantees}

In this section, we prove the different properties \refalg{stm} provides.
First, we show that our STM design is weakly progressive and stricter serializable.
Then, we prove that, when $\clockOf{p}$ is local, \refalg{stm} is disjoint-access parallel.

\paragraph{(Weak-progress)}
A transaction executes under \emph{weak progressiveness} \cite{Guerraoui:2009}, or equivalently it is \emph{weakly progressive}, when it aborts only if it encounters a conflicting transaction.
By extension, an STM is weakly progressive when it only produces histories during which transactions are weakly-progressive.
We prove that this property holds for \refalg{stm}.

In \refalg{stm}, a transaction $T$ aborts either at \refline{stm:read:5}, \ref{line:alg:stm:read:7}, \ref{line:alg:stm:write:2}, \ref{line:alg:stm:write:5}, or \ref{line:alg:stm:try:2}.
We observe that in such a case either $T$ observes an item $x$ locked, or that the timestamp associated with $x$ has changed.
It follows that if $T$ aborts then it observes a conflict with a concurrent transaction.
From which we deduce that it is executing under weak progressiveness.

\paragraph{(Stricter serializability)}
Consider some run $\run$ of \refalg{stm}, and let $h$ be the history produced in \run.
At the light of its pseudo-code, every function defined in \refalg{stm} is wait-free.
As a consequence, we may consider without lack of generality that $h$ is complete, i.e., every transaction executed in $h$ terminates with either a commit or an abort event.
In what follows, we define that $\ll_h$ as the order in which writes to the object locations are linearized in $\rho$.
We first prove that $<$ is acyclic for this definition of $\ll_h$.
Then, we show that, if a transaction does not exhibit any unfair binding, then it observes a strictly consistent snapshot.
For some transaction, we shall note $\clockOf{T_i}_f$ the final value of $\clockOf{T}$.

\begin{proposition}
  \labprop{stm:1}
  Consider two transactions $T_i$ and $T_{j \neq i}$ in $h$.
  If either $T_i \depends T_j$ or  $x_j \ll_h x_i$ holds, then $\clockOf{T_i}_f \geq \clockOf{T_j}_f$ is true.
  In addition, if transaction $T_i$ commits then the ordering is strict, i.e., $\clockOf{T_i}_f > \clockOf{T_j}_f$.
\end{proposition}

\begin{proof}

  In each of the two cases, we prove that $\clockOf{T_i}_f \geq \clockOf{T_i}_f$ holds before transaction $T_i$ commits.
  \begin{itemize}
  \item[($T_i \depends T_j$)]
    Let $x$ be an object such that $r_i(x_j)$ occurs in $h$.
    Since transaction $T_i$ reads version $x_j$, transaction $T_j$ commits.
    We observe that $T_j$ writes version $x_j$ together with $\clockOf{T_j}_f$ at $\locationOf{x}$ when it commits (\refline{stm:commit:4}).
    As a consequence, when transaction $T_i$ returns version $x_i$ at \refline{stm:read:9}, it assigns $\clockOf{T_j}_f$ to $t$ before at \refline{stm:read:2}.
    The condition at \refline{stm:read:6} implies that either $\clockOf{T_i} \geq t$ holds, or a call to $\stmExtend{T_i,t}$ occurs.
    In the latter case, transactino $T_i$ executes \refline{stm:extend:5}, advancing its clock up to the value of $t$.
  \item[($x_j \ll_h x_i$)]
    By definition, relation $\ll_h$ forms a total order over all versions of $x$.
    Thus, we may reason by induction, considering that $x_i$ is immediately after $x_j$ in the order $\ll_h$.
    When $T_j$ returns from $w_j(x_j)$ at \refline{stm:write:6}, it holds a lock on $x$.
    This lock is released at \refline{stm:commit:5} after writing to $\locationOf{x}$.
    As $\ll_h$ follows the linearization order, $T_i$ executes \refline{stm:write:1} after $T_j$ wrote $(x_j, \clockOf{T_j}_f)$ to $\locationOf{x}$.
    Location $\locationOf{x}$ is not updated between $x_j$ and $x_i$.
    Hence, after$T_i$ executes \refline{stm:write:4}, $\clockOf{T_i} \geq \clockOf{T_j}$ holds.
  \end{itemize}
  Since a clock is monotonic, the relation holds forever.
  Then, if transaction $T_i$ commits, it must executes \refline{stm:commit:1}, leading to $\clockOf{T_i}_f > \clockOf{T_i}_f$.
\end{proof}

\begin{proposition}
  \labprop{stm:2}
  Consider two transactions $T_i$ and $T_{j \neq i}$ in $\committed{h}$ such that $T_i < T_j$.
  Then, transaction $T_i$ invokes $\stmCommitFunction$ before transaction $T_j$ in $h$.
\end{proposition}

\begin{proof}
  Assume that $T_i$ and $T_j$ conflict of some object $x$.
  We examine in order each of the four cases defining relation $<$.
  \begin{compactitem}
  \item ($T_i \hb_h T_j$)\\
    This case is immediate.
  \item ($\exists x : r_j(x_i) \in h$)\\
    Before committing, $T_j$ invokes \stmExtendFunction at \refline{stm:try:1}.
    Since $T_j$ commits in $h$, it should retrieve $(x_i,\msgAny)$ from $\locationOf{x}$ when executing \refline{stm:extend:2}.
    Hence, transaction $T_i$ has already exeecuted \refline{stm:commit:4} on object $x$.
    It follows that $T_i$ invokes $\stmCommitFunction$ before transaction $T_j$ in history $h$.
  \item ($\exists x : x_i \ll_h x_j$)\\
    By definition of $\ll_h$, the write of version $x_i$ is linearized before the write of version $x_j$ in $\rho$.
    After $T_i$ returns from $w_i(x_i)$, it owns a lock on object $x$ (\refline{stm:commit:4}).
    The object is then unlocked by transaction $T_i$ at \refline{stm:commit:5}.
    As a consequence, transaction $T_i$ takes a lock on object $x$ after $T_i$ invokes operation $\stmCommitFunction$.
    From which it follows that the claim holds.
  \item ($\exists x, T_k : x_k \ll_x x_j \land r_i(x_k)$)\\
    For the sake of contradiction, assume that $T_j$ invokes $\stmCommitFunction$ before $T_i$.
    When $T_j$ invokes $\stmCommitFunction$, it holds a lock on $x$.
    This lock is released at \refline{stm:commit:5} after version $x_j$ is written at location $\locationOf{x}$.
    %
    Then, consider the time at which $T_i$ invokes $\stmTryCommitFunction$.
    The call at \refline{stm:try:1} leads to fetching $\locationOf{x}$ at \refline{stm:extend:2}.
    Since $T_i$ reads version $x_k$ in $h$, a pair $(\clockOf{T_k}_f, x_k)$ is in $\readSetOf{T_i}$.
    From the definition of $\ll_h$ the write of $(\clockOf{T_k}_f, x_k)$ takes place before the write of version $(\clockOf{T_j}_f,x_j)$ in $\rho$.
    Hence, $\locationOf{x}$ does not contain anymore $(\clockOf{T_k}_f, x_k)$
    Applying \refprop{stm:1}, $T_i$ executes \refline{stm:extend:4} and aborts at \refline{stm:try:3}.
    Contradiction.
  \end{compactitem}
  
\end{proof}

\begin{proposition}
  \labprop{stm:3}
  History $h$ does not exhibit any RC-anti-dependencies ($h \notin \RCAD$)
\end{proposition}

\begin{proof}

  Consider $T_i$, $T_j$ and $T_k$ such that $r_i(x_k), w_j(x_j) \in h$, $x_k \ll_h x_j$ and $T_j$ commits before $T_i$.
  %
  When $T_i$ invokes $\stmTryCommit$, transaction $T_j$ is committed.
  Thus, when $T_i$ executes \refline{stm:try:1} to call $\stmExtendFunction$, it fetches $(x_j,\clockOf{T_j})$ from $\locationOf{x}$.
  On the other hand, $(x_k, \clockOf{T_k}_f) \in \readSetOf{T_i}$ holds at that time.
  From \refprop{stm:1}, $\clockOf{T_j}_f \neq \clockOf{T_k}_f$.
  It follows that the test at \refline{stm:extend:3} fails, leading $T_j$ to abort at \refline{stm:try:2}.  
  
\end{proof}

\begin{theorem}
  \labtheo{spser}
  History $h$ belongs to $\SPSER$.
\end{theorem}

\begin{proof}
  \refprop{stm:2} tells us that if $T_i < T_j$ holds then $T_i$ commits before $T_j$.
  It follows that the strict serialization graph $(\committed{h},<)$ is acyclic.

  Let us now turn our attention to the second property of $\SPSER$.
  To this end, assume that a transaction $T_i$ aborts in $h$.
  For the sake of contradiction, consider that $T_i$ exhibits fair bindings and yet observes a non-strictly consistent snapshot.

  Applying the definition given in \refsection{criteria:model}, there exist transactions $T_j$ and $T_{k \neq j}$ such that $T_i \depends T_j$, $r_i(x_k)$ occurs in $h$ and $x_k \ll_h x_j$.  
  If $T_j \ll_{h} T_i$, then transaction $T_i$ cannot observe version $x_k$.
  Consequently, transaction $T_j$ is concurrent to $T_i$.
  In addition, there exists a transaction $T_l$ and some object $y$ such that $T_i$ performs $r_i(y_l)$ and $T_l \depends T_j$.

  Relation $<$ is acyclic, thus $x_k \neq y_l$ holds.  
  It remains to consider the following two cases:
  \begin{compactitem}
  \item ($r_i(y_l) \hb_h r_i(x_k)$)\\
    From \refprop{stm:2} and $T_l \depends T_j$, transaction $T_j$ is committed at the time $T_i$ reads object $x$.
    Contradiction.
  \item ($r_i(x_k) \hb_h r_i(y_l)$)\\
    We first argue that the timestamp fetches from $\locationOf{y}$ at the time $T_i$ executes \refline{stm:read:2} is greater than $\clockOf{T_i}$.
    \begin{proof}
      First of all, observe that $T_j$ is not committed at the time $T_i$ reads object $x$ (since $x_{k} \ll_h x_{j}$ holds).
      Denoting $q$ the process that executes $T_j$, $\clockOf{q} < \clockOf{T_j}_f$ is true when $T_i$ begins its execution at \refline{stm:start:1}.
      %%    
      Since transaction $T_l$ is concurrent to $T_i$ and $r_i(y_l)$ occurs, $T_i$ is bound to $T_l$ on $y$.
      Assume some object $z$ read by $T_i$ before $y$.
      Because the binding of $T_i$ to $T_l$ is fair, $T_l$ (or one of its dependencies) accesses $z$.
      As $h \in \RCAD$, if some transaction $T_r$ updates to version $z_r$ the object $z$ read by $T_i$ this transaction cannot be concurrent to $T_l$.
      Hence, applying \refprop{stm:1}, $\clockOf{T_r}_f < \clockOf{T_l}_f$ holds.
      This leas to the fact that $\clockOf{T_i} < \clockOf{T_l}_f$ at the time $T_i$ invokes \stmReadFunction on $x$.
      k
    \end{proof}
    Consequently, transaction $T_i$ invokes \stmExtendFunction at \refline{stm:refline:6}.
    Transaction $T_j$ is committed at that time.
    Since $T_j \depends T_l$, $T_j$ is also commited.
    Following a reasoning similar to the one given in \refprop{stm:3}, the test at \refline{stm:extend:3} fails.
  \end{compactitem}
\end{proof}

\paragraph{(Disjoint-access parallelism)}
The logical clocks used in \refalg{stm} can be shared or local to each process.
When they are are local, function $\clockOf{p}$ becomes injective.
Consider such a scenario and two transactions $T_i$ and $T_j$ that do not access on a common object.
If $T_i$ and $T_j$ are executed by distinct processes, then they do not contend on some base object in \refalg{stm}.
It follows that \refalg{stm} is strictly disjoint-access parallel.



\subsection{Details}
\labsection{stm:details}

In \refalg{stm}, each object $x$ has a \emph{location} in the shared memory, denoted $\locationOf{x}$.
This location stores a pair $(t,d)$, where $t \in \naturalSet$ is a \emph{timestamp}, and $d$ is the actual content of $x$ as seen by transactions.
For simplicity, we shall name hereafter a pair $(t,d)$ a \emph{version} of object $x$.
Since the location of object $x$ is unique, a single version of object $x$ may exist at a time in the memory.
As usual, we asume some transaction $\transInit$ that intializes for every object $x$ the location $\locationOf{x}$ to $(0,\bot)$.
Furthermore, we consider that each object $x$ is atomic.

\refalg{stm} associates each object with a lock.
To manipulate the lock-related functions of object $x$, 
a process $p$ employs appropriately the functions $\lock{x}$, $\isLocked{x}$ and $\unlock{x}$.

For every transaction $T$ submitted to the system, \refalg{stm} maintains three local data structures:
\begin{inparaenum}[]
\item $\clockOf{T}$ is the logical clock of transaction $T$,
\item $\readSetOf{T}$ is a map that contains its \emph{read set}, and 
\item $\writeSetOf{T}$ is another map that stores the \emph{write set} of $T$.
\end{inparaenum}
\refalg{stm} updates incrementally $\readSetOf{T}$ and $\writeSetOf{T}$ over the course of the execution.
The read set serves to check that the view of the shared memory, or \emph{snpashot}, seen by the transaction is consistent.
The write set buffers updates.
In detail, the execution of a transaction $T$ proceeds as follows.

\begin{itemize}
\item[-] %
  When $T$ starts its execution, \refalg{stm} initializes $\clockOf{T}$ to the value of $\clockOf{p}$, then both $\readSetOf{T}$ and $\writeSetOf{T}$ to $\emptySet$ (\reflines{stm:start:1}{stm:start:3}).
\item[-] %
  When $T$ accesses a shared object $x$, if $x$ was previously written, its value is returned (\refline{stm:read:1}).
  Otherwise, \refalg{stm} fetches atomically the version $(d,t)$, as seen in location $\locationOf{x}$.
  Then, the algorithm checks that 
  \begin{inparaenum}
  \item no lock is held on $x$, and 
  \item in case $x$ was previously accessed, that $T$ observes the same version.
  \end{inparaenum}
  If one of these two conditions fails, \refalg{stm} aborts transaction $T$ (\refline{stm:read:5}).
  The algorithm then checks that the timestamp $t$ associated to the content $d$ is smaller than the clock of $T$.
  In case this does not hold (\refline{stm:read:6}), \refalg{stm} tries extending the snapshot of $T$ by calling function $\stmExtend{}$.
  This function returns $\true$ when the versions previously read by $T$ are still valid.
  In which case, $\clockOf{T}$ is updated to the value $t$.
  If \refalg{stm} succeeds in extending (if needed) the snapshot of $T$, $d$ is returned and the read set of $T$ updated accordingly;
  otherwise transaction $T$ is aborted (\refline{stm:read:7}).
\item[-] %
  Upon executing a write request on behalf of $T$ to some object $x$, \refalg{stm} takes the lock associated with $x$ (\refline{stm:write:1}), and in case of success, it buffers the update value $d$ in $\writeSetOf{T}$ (\refline{stm:write:6}).
  The timestamp $t$ of $x$ at the time \refalg{stm} takes the lock serves two purposes.
  First, \refalg{stm} checks that $t$ is lower than the current clock of $T$, and if not $T$ is extended (\refline{stm:write:4}).
  Second, it is saved in \writeSetOf{T} to ensure that at commit time the timestamp of the version of $x$ written by $T$ is greater than $t$.
\item[-] %
  When $T$ requests to commit, \refalg{stm} certifies the read set by calling function $\stmExtend{}$ with the clock of $T$ (\refline{stm:try:1}).
  If this test succeeds, transaction $T$ commits (\reflines{stm:commit:1}{stm:commit:6}).
  In such a case, \clockOf{T} ticks to reach its final value (\refline{stm:commit:1}).
  By construction, this value is greater than the clock of process $p$ at the time transaction $T$ started (\refline{stm:start:1}), as well as all the timestamps of all the versions read or written by $T$ (\reflinestwo{stm:read:5}{stm:write:4}).
  \refalg{stm} updates the clock of $p$ with the final value of \clockOf{T} (\refline{stm:commit:2}), then it updates the items written by $T$ with their novel versions (\refline{stm:commit:4}).
\end{itemize}

\refalg{stm} replaces the global clock usually employed in STM architectures with a locality-aware clock.
When \clockOf{p} is local to each process, \refalg{stm} ensures strict disjoint access parallelism (DAP) \cite{Attiya2015}.
More formally, this means that two transaction access concurrently a bae object of the implementation only if they contend on some shared object.
Provided the workload is parallel, DAP ensures the scalability of \refalg{stm}.
We assess empirically this claim in \refsection{evaluation}.

On the other hand, if processes need to synchronize too often, maintaining consistency among the various clocks is expensive.
In this situation, it might be of interest to find a compromise between the cost of cache coherency and the need for synchronization.
For instance, in a NUMA architecture, \refalg{stm} may assign a common clock per hardware socket.
Upon a call to $\clockOf{p}$, the algortihm returns the clock defined for the socket in which the processor executing process $p$ resides.

\subsection{Guarantees}
\labsection{stm:guarantees}

In this section, we prove the different properties \refalg{stm} provides.
First, we show that our STM design is weakly progressive and stricter serializable.
Then, we prove that, when $\clockOf{p}$ is local, \refalg{stm} is disjoint-access parallel.

\paragraph{(Weak-progress)}
A transaction executes under \emph{weak progressiveness} \cite{Guerraoui:2009}, or equivalently it is \emph{weakly progressive}, when it aborts only if it encounters a conflicting transaction.
By extension, an STM is weakly progressive when it only produces histories during which transactions are weakly-progressive.
We prove that this property holds for \refalg{stm}.

In \refalg{stm}, a transaction $T$ aborts either at \refline{stm:read:5}, \ref{line:alg:stm:read:7}, \ref{line:alg:stm:write:2}, \ref{line:alg:stm:write:5}, or \ref{line:alg:stm:try:2}.
We observe that in such a case either $T$ observes an item $x$ locked, or that the timestamp associated with $x$ has changed.
It follows that if $T$ aborts then it observes a conflict with a concurrent transaction.
From which we deduce that it is executing under weak progressiveness.

\paragraph{(Stricter serializability)}
Consider some run $\run$ of \refalg{stm}, and let $h$ be the history produced in \run.
At the light of its pseudo-code, every function defined in \refalg{stm} is wait-free.
As a consequence, we may consider without lack of generality that $h$ is complete, i.e., every transaction executed in $h$ terminates with either a commit or an abort event.
In what follows, we define that $\ll_h$ as the order in which writes to the object locations are linearized in $\rho$.
We first prove that $<$ is acyclic for this definition of $\ll_h$.
Then, we show that, if a transaction does not exhibit any unfair binding, then it observes a strictly consistent snapshot.
For some transaction, we shall note $\clockOf{T_i}_f$ the final value of $\clockOf{T}$.

\begin{proposition}
  \labprop{stm:1}
  Consider two transactions $T_i$ and $T_{j \neq i}$ in $h$.
  If either $T_i \depends T_j$ or  $x_j \ll_h x_i$ holds, then $\clockOf{T_i}_f \geq \clockOf{T_j}_f$ is true.
  In addition, if transaction $T_i$ commits then the ordering is strict, i.e., $\clockOf{T_i}_f > \clockOf{T_j}_f$.
\end{proposition}

\begin{proof}

  In each of the two cases, we prove that $\clockOf{T_i}_f \geq \clockOf{T_i}_f$ holds before transaction $T_i$ commits.
  \begin{itemize}
  \item[($T_i \depends T_j$)]
    Let $x$ be an object such that $r_i(x_j)$ occurs in $h$.
    Since transaction $T_i$ reads version $x_j$, transaction $T_j$ commits.
    We observe that $T_j$ writes version $x_j$ together with $\clockOf{T_j}_f$ at $\locationOf{x}$ when it commits (\refline{stm:commit:4}).
    As a consequence, when transaction $T_i$ returns version $x_i$ at \refline{stm:read:9}, it assigns $\clockOf{T_j}_f$ to $t$ before at \refline{stm:read:2}.
    The condition at \refline{stm:read:6} implies that either $\clockOf{T_i} \geq t$ holds, or a call to $\stmExtend{T_i,t}$ occurs.
    In the latter case, transactino $T_i$ executes \refline{stm:extend:5}, advancing its clock up to the value of $t$.
  \item[($x_j \ll_h x_i$)]
    By definition, relation $\ll_h$ forms a total order over all versions of $x$.
    Thus, we may reason by induction, considering that $x_i$ is immediately after $x_j$ in the order $\ll_h$.
    When $T_j$ returns from $w_j(x_j)$ at \refline{stm:write:6}, it holds a lock on $x$.
    This lock is released at \refline{stm:commit:5} after writing to $\locationOf{x}$.
    As $\ll_h$ follows the linearization order, $T_i$ executes \refline{stm:write:1} after $T_j$ wrote $(x_j, \clockOf{T_j}_f)$ to $\locationOf{x}$.
    Location $\locationOf{x}$ is not updated between $x_j$ and $x_i$.
    Hence, after$T_i$ executes \refline{stm:write:4}, $\clockOf{T_i} \geq \clockOf{T_j}$ holds.
  \end{itemize}
  Since a clock is monotonic, the relation holds forever.
  Then, if transaction $T_i$ commits, it must executes \refline{stm:commit:1}, leading to $\clockOf{T_i}_f > \clockOf{T_i}_f$.
\end{proof}

\begin{proposition}
  \labprop{stm:2}
  Consider two transactions $T_i$ and $T_{j \neq i}$ in $\committed{h}$ such that $T_i < T_j$.
  Then, transaction $T_i$ invokes $\stmCommitFunction$ before transaction $T_j$ in $h$.
\end{proposition}

\begin{proof}
  Assume that $T_i$ and $T_j$ conflict of some object $x$.
  We examine in order each of the four cases defining relation $<$.
  \begin{compactitem}
  \item ($T_i \hb_h T_j$)\\
    This case is immediate.
  \item ($\exists x : r_j(x_i) \in h$)\\
    Before committing, $T_j$ invokes \stmExtendFunction at \refline{stm:try:1}.
    Since $T_j$ commits in $h$, it should retrieve $(x_i,\msgAny)$ from $\locationOf{x}$ when executing \refline{stm:extend:2}.
    Hence, transaction $T_i$ has already exeecuted \refline{stm:commit:4} on object $x$.
    It follows that $T_i$ invokes $\stmCommitFunction$ before transaction $T_j$ in history $h$.
  \item ($\exists x : x_i \ll_h x_j$)\\
    By definition of $\ll_h$, the write of version $x_i$ is linearized before the write of version $x_j$ in $\rho$.
    After $T_i$ returns from $w_i(x_i)$, it owns a lock on object $x$ (\refline{stm:commit:4}).
    The object is then unlocked by transaction $T_i$ at \refline{stm:commit:5}.
    As a consequence, transaction $T_i$ takes a lock on object $x$ after $T_i$ invokes operation $\stmCommitFunction$.
    From which it follows that the claim holds.
  \item ($\exists x, T_k : x_k \ll_x x_j \land r_i(x_k)$)\\
    For the sake of contradiction, assume that $T_j$ invokes $\stmCommitFunction$ before $T_i$.
    When $T_j$ invokes $\stmCommitFunction$, it holds a lock on $x$.
    This lock is released at \refline{stm:commit:5} after version $x_j$ is written at location $\locationOf{x}$.
    %
    Then, consider the time at which $T_i$ invokes $\stmTryCommitFunction$.
    The call at \refline{stm:try:1} leads to fetching $\locationOf{x}$ at \refline{stm:extend:2}.
    Since $T_i$ reads version $x_k$ in $h$, a pair $(\clockOf{T_k}_f, x_k)$ is in $\readSetOf{T_i}$.
    From the definition of $\ll_h$ the write of $(\clockOf{T_k}_f, x_k)$ takes place before the write of version $(\clockOf{T_j}_f,x_j)$ in $\rho$.
    Hence, $\locationOf{x}$ does not contain anymore $(\clockOf{T_k}_f, x_k)$
    Applying \refprop{stm:1}, $T_i$ executes \refline{stm:extend:4} and aborts at \refline{stm:try:3}.
    Contradiction.
  \end{compactitem}
  
\end{proof}

\begin{proposition}
  \labprop{stm:3}
  History $h$ does not exhibit any RC-anti-dependencies ($h \notin \RCAD$)
\end{proposition}

\begin{proof}

  Consider $T_i$, $T_j$ and $T_k$ such that $r_i(x_k), w_j(x_j) \in h$, $x_k \ll_h x_j$ and $T_j$ commits before $T_i$.
  %
  When $T_i$ invokes $\stmTryCommit$, transaction $T_j$ is committed.
  Thus, when $T_i$ executes \refline{stm:try:1} to call $\stmExtendFunction$, it fetches $(x_j,\clockOf{T_j})$ from $\locationOf{x}$.
  On the other hand, $(x_k, \clockOf{T_k}_f) \in \readSetOf{T_i}$ holds at that time.
  From \refprop{stm:1}, $\clockOf{T_j}_f \neq \clockOf{T_k}_f$.
  It follows that the test at \refline{stm:extend:3} fails, leading $T_j$ to abort at \refline{stm:try:2}.  
  
\end{proof}

\begin{theorem}
  \labtheo{spser}
  History $h$ belongs to $\SPSER$.
\end{theorem}

\begin{proof}
  \refprop{stm:2} tells us that if $T_i < T_j$ holds then $T_i$ commits before $T_j$.
  It follows that the strict serialization graph $(\committed{h},<)$ is acyclic.

  Let us now turn our attention to the second property of $\SPSER$.
  To this end, assume that a transaction $T_i$ aborts in $h$.
  For the sake of contradiction, consider that $T_i$ exhibits fair bindings and yet observes a non-strictly consistent snapshot.

  Applying the definition given in \refsection{criteria:model}, there exist transactions $T_j$ and $T_{k \neq j}$ such that $T_i \depends T_j$, $r_i(x_k)$ occurs in $h$ and $x_k \ll_h x_j$.  
  If $T_j \ll_{h} T_i$, then transaction $T_i$ cannot observe version $x_k$.
  Consequently, transaction $T_j$ is concurrent to $T_i$.
  In addition, there exists a transaction $T_l$ and some object $y$ such that $T_i$ performs $r_i(y_l)$ and $T_l \depends T_j$.

  Relation $<$ is acyclic, thus $x_k \neq y_l$ holds.  
  It remains to consider the following two cases:
  \begin{compactitem}
  \item ($r_i(y_l) \hb_h r_i(x_k)$)\\
    From \refprop{stm:2} and $T_l \depends T_j$, transaction $T_j$ is committed at the time $T_i$ reads object $x$.
    Contradiction.
  \item ($r_i(x_k) \hb_h r_i(y_l)$)\\
    We first argue that the timestamp fetches from $\locationOf{y}$ at the time $T_i$ executes \refline{stm:read:2} is greater than $\clockOf{T_i}$.
    \begin{proof}
      First of all, observe that $T_j$ is not committed at the time $T_i$ reads object $x$ (since $x_{k} \ll_h x_{j}$ holds).
      Denoting $q$ the process that executes $T_j$, $\clockOf{q} < \clockOf{T_j}_f$ is true when $T_i$ begins its execution at \refline{stm:start:1}.
      %%    
      Since transaction $T_l$ is concurrent to $T_i$ and $r_i(y_l)$ occurs, $T_i$ is bound to $T_l$ on $y$.
      Assume some object $z$ read by $T_i$ before $y$.
      Because the binding of $T_i$ to $T_l$ is fair, $T_l$ (or one of its dependencies) accesses $z$.
      As $h \in \RCAD$, if some transaction $T_r$ updates to version $z_r$ the object $z$ read by $T_i$ this transaction cannot be concurrent to $T_l$.
      Hence, applying \refprop{stm:1}, $\clockOf{T_r}_f < \clockOf{T_l}_f$ holds.
      This leas to the fact that $\clockOf{T_i} < \clockOf{T_l}_f$ at the time $T_i$ invokes \stmReadFunction on $x$.
      k
    \end{proof}
    Consequently, transaction $T_i$ invokes \stmExtendFunction at \refline{stm:refline:6}.
    Transaction $T_j$ is committed at that time.
    Since $T_j \depends T_l$, $T_j$ is also commited.
    Following a reasoning similar to the one given in \refprop{stm:3}, the test at \refline{stm:extend:3} fails.
  \end{compactitem}
\end{proof}

\paragraph{(Disjoint-access parallelism)}
The logical clocks used in \refalg{stm} can be shared or local to each process.
When they are are local, function $\clockOf{p}$ becomes injective.
Consider such a scenario and two transactions $T_i$ and $T_j$ that do not access on a common object.
If $T_i$ and $T_j$ are executed by distinct processes, then they do not contend on some base object in \refalg{stm}.
It follows that \refalg{stm} is strictly disjoint-access parallel.


%\section{Extensions}
\labsection{ext}

This section present seeveral extensions for \refalg{stm}.
First, we explain how to attain opacity, a stricter consistency model than strict serializability.
Then, we present a mechanism to adapt clocks dynamically to the workload, in order to find dynamically a sweet point between the need for synchronization and the cost of cache coherence.

\subsection{Attaining opacity}
\labsection{ext:opacity}

\paragraph{(A condition on workload.)}
Consider three transactions $T_a=[r(z);r(x);r(y)]$, $T_2=[w(x)]$ and $T_3=[r(x);w(y)]$.
In addition, assume a initial state in which register $z$ holds a version $z_{100}$, while both $x$ and $y$ are in their initial state.
In this history, $T_a$ fails to construct a consistent snapshot: it reads version $y_2$, yet do not observe $x_1$.

\begin{figure}[!h]
  \centering
  \fontsize{8}{11}\selectfont
  \begin{tikzpicture}[scale=0.77]
    \node at (10,.3) {$(h_1)$};
    
    \node at (0.2,1.8) {$T_1$};
    \node at (0.2,1) {$T_2$};
    
    \path[->] (0.5,1) edge (10,1);
    \path[->] (0.5,1.8) edge (10,1.8);
    
    \path[callA] (1.5,1.8) edge (2.5,1.8);
    \path[callA,->] (1.5,2) edge (1.5,1.8);
    \path[callA,->] (2.5,1.8) edge (2.5,2);
    \node at (1.5,2.2) {$r_1(x_0)$}; 
    \node at (2.5,2.2) {};

    \path[callA] (4,1.8) edge (5,1.8);
    \path[callA,->] (4,2) edge (4,1.8);
    \path[callA,->] (5,1.8) edge (5,2);
    \node at (4,2.2) {$r_1(y_2)$};
    \node at (5,2.2) {};

    \path[callA] (7,1.8) edge (8,1.8);
    \path[callA,->] (7,2) edge (7,1.8);
    \path[callA,->] (8,1.8) edge (8,2);
    \node at (7,2.2) {$\flagAbort$};
    \node at (8,2.2) {};

    \path[callB] (2.5,1) edge (3.5,1);
    \path[callB,->] (2.5,1.2) edge (2.5,1);
    \path[callB,->] (3.5,1) edge (3.5,1.2);
    \node at (2.5,1.4) {$w_2(x_2)$};
    \node at (3.5,1.4) {};

    \path[callB] (4.5,1) edge (5.5,1);
    \path[callB,->] (4.5,1.2) edge (4.5,1);
    \path[callB,->] (5.5,1) edge (5.5,1.2);
    \node at (4.5,1.4) {$w_2(y_2)$};
    \node at (5.5,1.4) {};

    \path[callB] (7,1) edge (8,1);
    \path[callB,->] (7,1.2) edge (7,1);
    \path[callB,->] (8,1) edge (8,1.2);
    \node at (7,1.4) {$\flagCommit$};
    \node at (8,1.4) {};
    
    \pgfresetboundingbox
    \clip[use as bounding box] (0,.7) rectangle (10,2);
  \end{tikzpicture}
\end{figure}


When clocks are local, history $h_1$ is admissible \refalg{stm}
This comes from the fact that when $T_a$ reads version $z_{100}$, the clock of $z$ has been increased to a value higher than the clock associated with $y_1$.
Hence, when $T_a$ access $y$, it does not extend and check the content of $x$.

One solution to this problem consists in calling $\stmExtend{}$ everytime a new item is read (similarly to \cite{}).
However, as underlined in \refsection{introduction}, this raises time complexity to $O(r^2)$, where $r$ is the amount of reads in the transaction.
Instead, we make the observation that if $z$ is not read in $T_2$ then it is not causaly-related to $y$.
Our idea is to require that the path to $y$ solely contains causaly-related registrs.

With more details,

\begin{itemize}
\item[$\mathcal{C}$]
  When a transaction $T_i$ reads a register $x$ then, for every register $y$ read but not written before $x$, if a concurrent transaction $T_j$ writes to register $x$, $y \in \readSetOf{T_j}$ holds.
\end{itemize}

The proposition that follows proves that if condition $\mathcal{C}$ holds, \refalg{stm} produced opaque histories.

\begin{theorem}
  \labprop{ext:2}
  \refalg{stm}' is opaque.
\end{theorem}

\begin{proof}

  Clearly \refalg{stm}' attains strict serializability by immediate reduction to the original algorithm.
  To show that a transaction always sees a consistent snapshot we proceed by contradiction, as in \refprop{ext:1}.
  %
  Let us assume that some transaction $T_a \in h$ does not observe a consistent snapshot.
  This means that transaction $T_a$
  \begin{inparaenum}
  \item reads a version $x_j$ of some register $x$,
  \item depends on some transaction $T_k$ which writes version $x_k$ in $h$, and
  \item $x_j$ is followed by version $x_k$ in the version order $\ll_{h,x}$.
  \end{inparaenum}
  We observe that $r_a(x_j)$ precedes $w_k(x_k)$ in $h$, otherwise $T_a$ must abort before reading $x$, or reads instead version $x_k$.
  Applying \refprop{} to transction $T_a \depends T_k$, there exist a write $w_k(z_k)$ and a read $r_a(y_l)$ such that
  \begin{inparaenum}
  \item $w_k(z_k) \hb_h r_a(y_l)$, and
  \item $\clockOf{y_l} \geq \clockOf{T_k}$.
  \end{inparaenum}  
  If $r_a(y_l) <_h r_a(x_j)$ holds, then $T_a$ aborts before return $x_j$ as transaction $T_k$ still holds the lock on register $x$.
  In the converse case, as $\clockOf{x_j} < \clockOf{x_k}$ holds, transaction $T_a$ aborts before returning $y_l$ when executing $\stmExtend{}$.  
    
\end{proof}

From a practical point of view, we now argue that the above algorithm avoids most of the time a complexity in $O(r^2)$.
To this end, let us conider some transaction $T$ reading two versions $x_i$ and $y_j$ of objects $x$ and $y$.
In the common case, it is expected that there is no concurrent transactions to $T$.
Thus,
\begin{inparaenum}
\item if $x$ and $y$ are bound due to some state invariant, $x_i$ and $y_j$ have the same clocks; and
\item if $x_i$ was updated causally before $y_j$, it suffices that $T$ reads $y_j$ before its causal dependency $x_i$.
\end{inparaenum}
Item (ii) is generally true in the case of a referential integrity linking $y$ to $x$ (i.e., $y$ stores a reference to $x$).


\paragraph{(Global clock.)}
When processes use a global clock, \refalg{stm} is idenrtical to the original code of TinySTM.
This means that each transaction, even if it aborts, observes a strictly consistent snapshot.
Such a property, together with \reftheo{ss}, leads to the fact that \refalg{stm} is opaque.
Before proving this claim, we first recall the notion of consisstent snapshot.

A transaction $T_i$ \emph{depends on} a transaction $T_j$ when $T_i$ reads a version written by $T_j$, or such a relation holds transitively.
Thus, a transaction observes a \emph{consistent snapshot} \cite{Chan:1985} when it never misses the effects of some transaction it depends on.
In other words, transaction $T_i$ in a history $h$ observes a consistent snapshot when
\begin{inparaenum}[\em(i)]
\item $T_i$ reads version $x_j$,
\item $T_k$ writes version $x_k$, and 
\item $T_i$ depends on $T_{k}$,
\end{inparaenum}
then version $x_k$ is followed by version $x_j$.
Formally, $((r_i(x_j) \in h) \land (w_k(x_k) \in h) \land (T_i \depends T_k)) \implies (x_k \ll_{h,x} x_j)$, where $\ll_{h,x}$ is the version order defined by history $h$.

\begin{proposition}
  \labprop{ext:1}
  If $\clockOf{p}$ is global for every process $p$, then for every transaction $T$, $\readSetOf{T}$ is always a strictly consistent snapshot.
\end{proposition}

\begin{proof}
  (By contradiction.)
  
  Let us consider a history $h$ produced by \refalg{stm} during which some transaction $T$
  \begin{inparaenum}[\em(i)]
  \item reads version $x_j$ of register $x$,
  \item depends on some transaction $T_k$ that writes version $x_k$, and
  \item $x_j$ is followed by version $x_k$ in the version order $\ll_{h,x}$.
  \end{inparaenum}

  As $T_i$ depends on $T_k$, $T_i$ reads some version $y_l$ of a register $y$, with $T_l$ depending on $T_k$.
  Thus, at the time, $T$ reads $y_l$, register $x$ is at least in version $x_k$.
  It followed that $T$ must read register $x$ before it read register $y$.
  We have two cases to consider here.
  First, assume that the clock of $T$ is higher than the version of $y_l$.
  In such a case, as the time is global, $T$ must have read from a transaction that committed after $T_k$.
  Because $y_l$ is the first item then.
  Otherwose, $T$ must execute an extend operation and fails reading version $x_j$.  
\end{proof}

Let us observe that as each read when it returns ensure that the snapshot read so far is consistent, when the clock is global the algorithm does not call \stmExtend{} at \refline{stm:try:1}.

\subsection{Conjoining clocks}
\labsection{stm:conjoining}

Local clocks leverage parallelism but it is expected that a global clock performs better under contended workloads.
\refsection{evaluation} confirms this expectation in practice.

To have the best of both worlds, we propose a mechanism that allows switching from one clock to another.
This mechanism is named \emph{conjoining clocks}.
In practice, we are interested in conjoining a global real-time clock together with a local logical clock.

With more details, conjoining two clocks $c$ and $\barc$ is defined as a new clock $c \times \barc$ with:
\begin{compactitem}
\item $\cread \equaldef [t \assign c.\cread];~\text{\textbf{return}}~t$
\item $\cadv(u) \equaldef [c.\cadv(u)]$
\item $\cswitch \equaldef [\barc.\cadv(c.\cread);~d \assign c;~c \assign \barc;~\barc \assign d]$
\end{compactitem}
where we mark using $[ \ldots ]$ the begin and end of a transaction.

\begin{proposition}
  If $c$ and $\barc$ two clocks, then $c \times \barc$ is a clock.
\end{proposition}

\begin{proof}  
  Let us consider that $\cadv(u) \hb_h \cread()$ occurs in some history $h$.
  Name $l$ the strictly serializable history equivalent to $h$.
  %
  First, assume that no operation $\cswitch$ occurs between the read and update events.
  It follows that $c.\cadv(u)$ happens before the read operation returns $c.\cread()$.
  Since $c$ is a clock $c.\cread() \geq u$ holds.
  %
  Otherwise, let us consider that a $\cswitch$ operation occurs in $l$.
  Note $c$ the clock before the switch and $c'$ after.
  Clearly, $c < c'$.
  
\end{proof}

In practice, we conjoin clocks with the help of a global lock.


\section{Evaluation}

\begin{frame}{$\SPSER$ transactional memory}
  Key algorithmic features:
  \begin{itemize}
  \item single version
  \item lock-based
  \item lazy snapshot
    \begin{itemize}
    \item[] when executing $r_i(x_j)$
    \item[] if $\clockOf{T_j} > \clockOf{T_i}$ then
    \item[] \hspace{1em} revalidate prior reads
    \end{itemize}        
  \item at commit time, $\clockOf{T_i} = \max \{ \clockOf{T_j} : T_i \depends T_j \} + 1$
  \end{itemize}
\end{frame}

\begin{frame}{Evaluation parameters}
  \begin{itemize}
  \item open-source C implementation (based on TinySTM)
  \item comparison with TinySTM (v1.0.5)
  \item TinySTM applications suite (bank, linked-list, red-black tree)
  \item AMD Opteron48
    \begin{itemize}
    \item a 48-cores machine with 256 GB of RAM.
    \item 4 dodeca-core AMD Opteron 6172 (8 NUMA nodes)
    \end{itemize}    
  \end{itemize}
\end{frame}

\begin{frame}{Bank benchmark}
  \begin{figure}
    \includegraphics[scale = 0.8]{results/intset/bank.pdf}
  \end{figure}
  Each account belongs to a branch.\\
  Locality = likeliness to pick an account in the thread-local branch.\\
  \smallskip
  In the above figure, Locality set to $0.8$ \\
\end{frame}
  
\begin{frame}{Bank benchmark}
  \begin{figure}
    \includegraphics[scale = 0.8]{results/bank-speedup/bank-speedup.pdf}
  \end{figure}
  Varying locality, using $48$ threads
\end{frame}

\begin{frame}{Other benchmarks}
  \begin{figure}
    \includegraphics[scale = 0.9]{results/intset/ll-rb.pdf}
  \end{figure}
  linked-list = randomly add/remove $x \in [-255,255]$ \\
  red-black tree = randomly add/remove $x \in [0,10^7]$ 
\end{frame}

\section{Related Work}
\labsection{relatedwork}

Transactional memory (TM) allows to design applications with the help of sequences of instructions that run in isolation one from another.
This paradigm greatly simplifies the programming of modern highly-parallel computer architectures.

% blocking TM
Ennals \cite{emals} suggests to build deadlock-free lock-based TMs rather than non-blocking ones.
Empirical evidences~\cite{dice06} and theoretical results~\cite{Guerraoui:2008,KuznetsovR15} support this claim.

% permissiveness
At first glance, it might be of interest that a TM design accepts all correct histories; a property named \emph{permissiveness} \cite{guerraoui08}.
Such TM algorithms need to track large dependencies \cite{Keidar:2009} and/or acquire locks for read operations~\cite{attiya2012single}.
However, both techniques are known to have a significant impact on performance.

% validation
Early TM implementations (such as DSTM~\cite{herlihy2003software}) validate all the prior reads when accessing a new object.
The complexity of this approach is quadratic in the number of objects read along the execution path.
A time-based TM avoids this effort by relying on the use a global clock to timestamp object versions.
Zhang et al. \cite{zhang2008commit} compare several such approaches, namely TL2~\cite{dice2006transactional}, LSA \cite{riegel2006lazy} and GCC\cite{spear2006conflict}.
They provide guidelines to reduce unnecessary validations and shorten the commit sequence.

% single version, invisible read
Multi-versioning~\cite{Fernandes:2011, Diegues:2014} brings a major benefit: allowing read-only transactions to complete.
This clearly boosts certain workloads but managing multiple versions has a non-negligible performance cost on the TM internals.
Similarly, invisible reads ensure that read operations do not contend in most cases.
However, such a technique limits progress or the consistency criteria satisfied by the TM \cite{Attiya:2009}.
In the case of \refalg{stm}, both read-only and updates transaction are certain to make progress only in the absence of contention.

% DAP
New challenges arise when considering multicore architectures and cache coherency strategies for NUMA architectures.
Clock contention~\cite{6121290} is one of them.
To avoid this problem, workloads as well as TM designs should take into account parallelism~\cite{Nguyen:2017}.
Chan et al \cite{6121290} propose to group threads into zones, and that each zone shares a clock and a clock table.
To timestamp a new version, the TL2C algorithm \cite{Avni:2008} tags it with a local counter together with the thread id.
Each thread stores a vector of the latest timestamp it encountered.
The algorithm preserves opacity by requiring that a transaction restarts if one of the vector entries is not up to date.

%% \vs{add PODC'10 NUMA-aware TM ~\cite{Lu:2010:BAN:1835698.1835713}}
%% \vs{add ProteusTM~\cite{didona2016proteustm}}
%% \vs{add ~\cite{Mohamedin:2016:DNC:2851141.2851189}}
%% \vs{cite ~\cite{nguyen2017scalable}}

% Maintaining Consistent Transactional States without a Global Clock, Avni & Shavit
% space suage: O(m) per thread.
% each transaction got a vector clock inddicating the ts at which other transactions started
% each location got a timestamp pair (x.ts,x.owner), where x.owner is the thread that wrote x last.
% upon reading x, T compare x.ts to  its ts and abort if x.ts[x.owner] > T.ts[x.owner]
% no forward reading possible, thus more false abprt:
%
% This algorithm is SSER but has two main disadvantages
% 1) 
% ``We argue that a TLC transaction will always fail if it attempts 
% to read a location that was written by some other transaction after it started.''
% example of spurious abort: w_p(x).c_p.r_q(x).w_q(y).c_p.r_{q'}(y).a_{q'}.r_{q'}(x).a_{q'}
% the progress property of this STM is very weak, namely
% if the transaction starts from a quescient state and it repeatdly executed and there is no concurrent transaction, then it commits.
% 2) abortion due to non-causally consistent snapshot
% our appraoch does not have this problem


%% With multiple CPUs, each with its own clock, its impossible to guarantee that the crystals
%% don’t differ after some amount of time even when initially set accurately. In practice, all clocks
%% counters will run at slightly different rates. This clock skew brings several problems that can occur
%% and several solutions as well, some more appropriate than others in certain contexts.
%% The Time Stamp Counter was once an excellent high-resolution, low-overhead way for a program to get CPU timing information. With the advent of multi-core/hyper-threaded CPUs, systems with multiple CPUs, and hibernating operating systems, the TSC cannot be relied upon to provide accurate results — unless great care is taken to correct the possible flaws: rate of tick and whether all cores (processors) have identical values in their time-keeping registers. There is no promise that the timestamp counters of multiple CPUs on a single motherboard will be synchronize
%% Relying on the TSC also reduces portability, as other processors may not have a similar feature

%% Thanks for all the inputs: Here's the conclusion for this discussion: The TSCs are synchronized at the initialization using a RESET that happens across the cores and processors in a multi processor/multi core system. And after that every Core is on their own. The TSCs are kept invariant with a Phase Locked Loop that would normalize the frequency variations and thus the clock variations within a given Core and that is how the TSC remain in sync across cores and processors.

%% https://stackoverflow.com/questions/10921210/cpu-tsc-fetch-operation-especially-in-multicore-multi-processor-environment
%% https://github.com/dterei/tsc

%% PCL theorem

% the alg. of shvit et al. exhibits a very weak progress property, namely if the transaction is retried infinitely often it eventually commits.

%%   Computer designs that exploit data locality with multiple cache levels, such as non-uniform memory architectures (NUMA), have to reduce the amount of global operations to improve application performance or take special care to address traffic congestions\cite{dashti2013traffic}.


\section{Conclusion}
\labsection{conclusion}

Transactional memory systems must handle a tradeoff between consistency and performance.
It is impractical to take into account all possible combinations of read and write conflicts, as it would lead to largely inefficient solutions.
%% For instance, accepting $\RCAD$ histories brings only a small performance benefits in the general case~\cite{hans16}.

%Inspired by line of research, 
This paper introduces a new consistency criterion, named stricter serializability ($\SPSER$).
Workloads executed under $\SPSER$ are opaque when the object graph forms a tree and transactions traverse it top-down.
We present an algorithm to attain this criterion together with a proof of its correctness.
Our evaluation based on a fully implemented prototype demonstrates that such an approach is very efficient in weakly-contended workloads.

% move from SSER to SSER+ (i.e., alg. transform) ?



{
  \footnotesize
  \bibliographystyle{abbrvnat}
  \bibliography{bib/nicolas,bib/psutra,bib/mshapiro,bib/paper}
}  

\end{document}
