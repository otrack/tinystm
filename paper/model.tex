\section{Model}
\labsection{model}

In what follows, we present the elements of our system model, as commonly found in textbooks (e.g., \cite{}).

Software transactional memory (STM) is a recent paradigm that allows multiple processes to access concurently a memory space.
Each process manipulates registers in the shared memory with the help of transactions.
When a process begins a new transactions, it calls function \stmBeginFunction.
% a tranasction may abort earlier ...
Then, it executes a sequence of \stmReadFunction and \stmWriteFunction operations on the registers according to some internal (not modeled) logic.
At the end of the execution, the process calls \stmTryCommitFunction to terminate the transaction, in which case the transaction may either commit ($\flagCommit$), or abort ($\flagAbort$).

A \emph{history} is a sequence of invocations and responses of the tranasctional operations by one or more processes.
It commonly is represented with timelines.
For instance, in history $h_1$ below, processes $p$ and $q$ access a data store that contains two registers ($x$ and $y$).

%% \input{fig-sample.tex}

A history induces a real-time order between transactions (denoted $\hb_h$).
This order holds between two transactions $T$ and $T'$ when $T$ terminates in $h$ before $T'$ starts.
A history also induces a per-process order that corresponds to the order in which processes invoke their transactions.
For instance in $h_1$, transaction ${\color{red}{T_2}}=w{x}$ precedes transaction ${\color{OliveGreen}{T_3}}=r(y)$ at process $q$.
This relation together with (${\color{blue}{T_1}} <_{h_1} {\color{OliveGreen}{T_3}}$) fully defines the real-time order in history $h_1$.

A version is the state of a register as produced by an update transaction.
For instance, when $T_i$ writes to some register $x$, an operation denoted hereafter $w(x_i)$, it creates a new version $x_i$ of $x$.
Versions allow to uniquely identify the state of a register as observed by a read operation, e.g., $r(x_i)$.
We shall say that $T_i$ \emph{read-write depends} on $T_j$ when $r_i(x_i)$ is in $h$.

Given some history $h$ and some register $x$, a \emph{version order} $\ll_{h,x}$ for $x$ in $h$ is a total order relation over the versions of $x$ in $h$.
A version order for $h$, denoted $\ll_{h}$ is the union of all the version orders for all registers.
For instance, in history $h_1$ above,  we may consider $x_0 \ll_{h_1} x_1$.

%% Our model is taken from ..
%% $\run$ to $h$
%% hist. seq. of events. inv/res.
%% def. $\hb_h$, $<_{h}$
%% def. wlog. version
%% def. $\depend$ 
