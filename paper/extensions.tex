\section{Extensions}
\labsection{ext}

This section present seeveral extensions for \refalg{stm}.
First, we explain how to attain opacity, a stricter consistency model than strict serializability.
Then, we present a mechanism to adapt clocks dynamically to the workload, in order to find dynamically a sweet point between the need for synchronization and the cost of cache coherence.

\subsection{Attaining opacity}
\labsection{ext:opacity}

\paragraph{(Global clock.)}
When processes use a global clock, \refalg{stm} is idenrtical to the original code of TinySTM.
This means that each transaction, even if it aborts, observes a strictly consistent snapshot.
Such a property, together with \reftheo{ss}, leads to the fact that \refalg{stm} is opaque.
Before proving this claim, we first recall the notion of consisstent snapshot.

A transaction $T_i$ \emph{depends on} a transaction $T_j$ when $T_i$ reads a version written by $T_j$, or such a relation holds transitively.
Thus, a transaction observes a \emph{consistent snapshot} \cite{Chan:1985} when it never misses the effects of some transaction it depends on.
In other words, transaction $T_i$ in a history $h$ observes a consistent snapshot when
\begin{inparaenum}[\em(i)]
\item $T_i$ reads version $x_j$,
\item $T_k$ writes version $x_k$, and 
\item $T_i$ depends on $T_{k}$,
\end{inparaenum}
then version $x_k$ is followed by version $x_j$.
Formally, $((r_i(x_j) \in h) \land (w_k(x_k) \in h) \land (T_i \depends T_k)) \implies (x_k \ll_{h,x} x_j)$, where $\ll_{h,x}$ is the version order defined by history $h$.

\begin{proposition}
  \labprop{ext:1}
  If $\clockOf{p}$ is global for every process $p$, then for every transaction $T$, $\readSetOf{T}$ is always a strictly consistent snapshot.
\end{proposition}

\begin{proof}
  (By contradiction.)
  
  Let us consider a history $h$ produced by \refalg{stm} during which some transaction $T$
  \begin{inparaenum}[\em(i)]
  \item reads version $x_j$ of register $x$,
  \item depends on some transaction $T_k$ that writes version $x_k$, and
  \item $x_j$ is followed by version $x_k$ in the version order $\ll_{h,x}$.
  \end{inparaenum}

  As $T_i$ depends on $T_k$, $T_i$ reads some version $y_l$ of a register $y$, with $T_l$ depending on $T_k$.
  Thus, at the time, $T$ reads $y_l$, register $x$ is at least in version $x_k$.
  It followed that $T$ must read register $x$ before it read register $y$.
  We have two cases to consider here.
  First, assume that the clock of $T$ is higher than the version of $y_l$.
  In such a case, as the time is global, $T$ must have read from a transaction that committed after $T_k$.
  Because $y_l$ is the first item then.
  Otherwose, $T$ must execute an extend operation and fails reading version $x_j$.  
\end{proof}

Let us observe that as each read when it returns ensure that the snapshot read so far is consistent, when the clock is global the algorithm does not call \stmExtend{} at \refline{stm:try:1}.

\paragraph{(Accurate tracking of causality.)}
Consider history $h_1$ described in \reffigure{} below.
In this history, transaction $T_a=[r(x);r(z);r(y)]$ reads $x_0$ and $y_1$.
Since transaction $T_1$ writes both register $x$ and $y$, transaction $T_a$ observe a fractured read \cite{}, and thus fails to construct a consistent snapshot.

%% \begin{figure}[!h]
  \centering
  \fontsize{8}{11}\selectfont
  \begin{tikzpicture}[scale=0.77]
    \node at (10,.3) {$(h_1)$};
    
    \node at (0.2,1.8) {$T_1$};
    \node at (0.2,1) {$T_2$};
    
    \path[->] (0.5,1) edge (10,1);
    \path[->] (0.5,1.8) edge (10,1.8);
    
    \path[callA] (1.5,1.8) edge (2.5,1.8);
    \path[callA,->] (1.5,2) edge (1.5,1.8);
    \path[callA,->] (2.5,1.8) edge (2.5,2);
    \node at (1.5,2.2) {$r_1(x_0)$}; 
    \node at (2.5,2.2) {};

    \path[callA] (4,1.8) edge (5,1.8);
    \path[callA,->] (4,2) edge (4,1.8);
    \path[callA,->] (5,1.8) edge (5,2);
    \node at (4,2.2) {$r_1(y_2)$};
    \node at (5,2.2) {};

    \path[callA] (7,1.8) edge (8,1.8);
    \path[callA,->] (7,2) edge (7,1.8);
    \path[callA,->] (8,1.8) edge (8,2);
    \node at (7,2.2) {$\flagAbort$};
    \node at (8,2.2) {};

    \path[callB] (2.5,1) edge (3.5,1);
    \path[callB,->] (2.5,1.2) edge (2.5,1);
    \path[callB,->] (3.5,1) edge (3.5,1.2);
    \node at (2.5,1.4) {$w_2(x_2)$};
    \node at (3.5,1.4) {};

    \path[callB] (4.5,1) edge (5.5,1);
    \path[callB,->] (4.5,1.2) edge (4.5,1);
    \path[callB,->] (5.5,1) edge (5.5,1.2);
    \node at (4.5,1.4) {$w_2(y_2)$};
    \node at (5.5,1.4) {};

    \path[callB] (7,1) edge (8,1);
    \path[callB,->] (7,1.2) edge (7,1);
    \path[callB,->] (8,1) edge (8,1.2);
    \node at (7,1.4) {$\flagCommit$};
    \node at (8,1.4) {};
    
    \pgfresetboundingbox
    \clip[use as bounding box] (0,.7) rectangle (10,2);
  \end{tikzpicture}
\end{figure}

 
When clocks are local, history $h_1$ is admissible \refalg{stm}
This comes from the fact that when $T_a$ reads register $z$, the clock of $z$ has been increased to a value higher than the clock associated with $y_1$.
Hence, when $T_a$ access $y$, it does not extend and check the content of $x$.

One solution to this problem consists in calling $\stmExtend{}$ everytime a new item is read (similarly to \cite{}).
However as underlined in \refsection{}, this leads to a time complexity of $O(r^2)$, where $r$ is the amount of reads in the transaction.
To reduce this complexity, we propose instead to check items for which the read timestamp is smaller than the new timestamp encountered.
In details, we introduce the following modifications to \refalg{stm}.
\begin{compactitem}
\item $\stmExtend{}$ is always calls at \refline{stm:read:5}, \refline{stm:write:4} and \refline{stm:try:1};
\item the computation at \reflines{stm:extend:2}{stm:extend:4} in $\stmExtend{}$ is executed only when $t' \leq t$ holds.
\end{compactitem}
We refer to this variation as \refalg{stm}' and proves below that it attains opacity.
% explain why this suffice.

\begin{theorem}
  \labprop{ext:2}
  \refalg{stm}' is opaque.
\end{theorem}

\begin{proof}

  Clearly \refalg{stm}' attains strict serializability by immediate reduction to the original algorithm.
  To show that a transaction always sees a consistent snapshot we proceed by contradiction, as in \refprop{ext:1}.
  %
  Let us assume that some transaction $T_a \in h$ does not observe a consistent snapshot.
  This means that transaction $T_a$
  \begin{inparaenum}
  \item reads a version $x_j$ of some register $x$,
  \item depends on some transaction $T_k$ which writes version $x_k$ in $h$, and
  \item $x_j$ is followed by version $x_k$ in the version order $\ll_{h,x}$.
  \end{inparaenum}
  We observe that $r_a(x_j)$ precedes $w_k(x_k)$ in $h$, otherwise $T_a$ must abort before reading $x$, or reads instead version $x_k$.
  Applying \refprop{} to transction $T_a \depends T_k$, there exist a write $w_k(z_k)$ and a read $r_a(y_l)$ such that
  \begin{inparaenum}
  \item $w_k(z_k) \hb_h r_a(y_l)$, and
  \item $\clockOf{y_l} \geq \clockOf{T_k}$.
  \end{inparaenum}  
  If $r_a(y_l) <_h r_a(x_j)$ holds, then $T_a$ aborts before return $x_j$ as transaction $T_k$ still holds the lock on register $x$.
  In the converse case, as $\clockOf{x_j} < \clockOf{x_k}$ holds, transaction $T_a$ aborts before returning $y_l$ when executing $\stmExtend{}$.  
    
\end{proof}

From a practical point of view, we now argue that the above algorithm avoids most of the time a complexity in $O(r^2)$.
To this end, let us conider some transaction $T$ reading two versions $x_i$ and $y_j$ of objects $x$ and $y$.
In the common case, it is expected that there is no concurrent transactions to $T$.
Thus,
\begin{inparaenum}
\item if $x$ and $y$ are bound due to some state invariant, $x_i$ and $y_j$ have the same clocks; and
\item if $x_i$ was updated causally before $y_j$, it suffices that $T$ reads $y_j$ before its causal dependency $x_i$.
\end{inparaenum}
Item (ii) is generally true in the case of a referential integrity linking $y$ to $x$ (i.e., $y$ stores a reference to $x$).

\subsection{Conjoining clocks}
\labsection{stm:conjoining}

Local clocks leverage parallelism but it is expected that a global clock performs better under contended workloads.
\refsection{evaluation} confirms this expectation in practice.

To have the best of both worlds, we propose a mechanism that allows switching from one clock to another.
This mechanism is named \emph{conjoining clocks}.
In practice, we are interested in conjoining a global real-time clock together with a local logical clock.

With more details, conjoining two clocks $c$ and $\barc$ is defined as a new clock $c \times \barc$ with:
\begin{compactitem}
\item $\cread \equaldef [t \assign c.\cread];~\text{\textbf{return}}~t$
\item $\cadv(u) \equaldef [c.\cadv(u)]$
\item $\cswitch \equaldef [\barc.\cadv(c.\cread);~d \assign c;~c \assign \barc;~\barc \assign d]$
\end{compactitem}
where we mark using $[ \ldots ]$ the begin and end of a transaction.

\begin{proposition}
  If $c$ and $\barc$ two clocks, then $c \times \barc$ is a clock.
\end{proposition}

\begin{proof}  
  Let us consider that $\cadv(u) \hb_h \cread()$ occurs in some history $h$.
  Name $l$ the strictly serializable history equivalent to $h$.
  %
  First, assume that no operation $\cswitch$ occurs between the read and update events.
  It follows that $c.\cadv(u)$ happens before the read operation returns $c.\cread()$.
  Since $c$ is a clock $c.\cread() \geq u$ holds.
  %
  Otherwise, let us consider that a $\cswitch$ operation occurs in $l$.
  Note $c$ the clock before the switch and $c'$ after.
  Clearly, $c < c'$.
  
\end{proof}

In practice, we conjoin clocks with the help of a global lock.

