\section{Related Work}
\labsection{relatedwork}

Software transactional memory systems must handle a tradeoff between consistency and performance.
It is however impractical to take into account all possible combinations of read and write conflicts, as it would lead to leargely inefficient solutions. 
%A conflict detection, taking into account all possible combinations of read and write conflicts would just not be efficient. 
Therefore, several mechanisms were introduced to reduce the effort of conflict detection and increase concurrency. 
DSTM~\cite{herlihy2003software} validates all previously accessed object when a new object is about to be opened. 
However, this approach does not scale, and its complexity is quadratic to the number of opened objects within a transaction.

Several time-based STM designs use a global version clock (e.g., Dice \cite{dice}, Riegel \cite{riegel2006lazy}, \cite{14})\vs{cannot infer the 3 previous references} to avoid the effort of incrementally validating the read set at commit time. 
TinySTM \cite{FelberFMR10} is a time-based STM that uses a lazy snapshot technique. 
On commit the TM assigns a timestamp from the global clock to the currently written objects. 
Any transaction constructs a snapshot of linearization points for ensuring consistency. 
By keeping a validity interval of timestamps, it can be verified whether a validation is really necessary. 

The authors of \cite{zhang2008commit} compare Transactional Locking II (TL2)~\cite{dice2006transactional}, Lazy Snapshot (LSS)\cite{riegel2006lazy} and Global Commit Counter (GCC)~\cite{spear2006conflict}. 
All of them require a new timestamp for each update transaction (total order).
Their approach perform unnecessary updates on the global counter thus leading to unnecessary validations. 
The authors provide several commit alternatives to solve these issues and reducing further unnecessary validations.

New challenges arise when considering multicore architectures and cache coherency strategies for NUMA architectures. 
Clock contention~\cite{6121290} is one of the major issues. 
In a multi-core system with several processors and separate caches, cache coherence messages are required on each update of the clock, which happens frequently. 
Transaction throughput is therefore limited.

To tackle these issues the authors of \cite{Avni:2008} introduce TL2C, which consider a distributed counter --- one per thread. 
The clock values propagate to a \emph{dlock} table with $n-1$ lock entries of different threads. 
During a validation phase, this cache is used, which may however lead to unnecessary aborts. 
Another issue is that the storage of the timestamps might not scale with the number of threads.

In \cite{6121290} the authors adapt TL2 to TrC-MC~\cite{chan2011trc} (transactional consistency for multicore), which groups threads into so-called \textit{zones}. 
Zones share a clock and a clock table.\vs{add a sentence to explain why this is relevant} 
To avoid unnecessary aborts, TrC-MC adopts a timestamp extension mechanism.
TrC-MC compares favourably against TL2 in terms of aborted transsactions.  
%Still the number of aborts are increased, while lower than caused by TL2C.

\vs{add ProteusTM~\cite{didona2016proteustm}}
\vs{add PODC'10 NUMA-aware TM ~\cite{Lu:2010:BAN:1835698.1835713}}
\vs{add ~\cite{Mohamedin:2016:DNC:2851141.2851189}}

\vs{Complete section with 2/3 sentences about how why NumaSTM is different }
%%Why is distributed TM not a relevant work
%%get some more from 612...
%%TODO, why is ours better?

% RWCounter (Lev & Moir) 
% this solution is still global 

% Maintaining Consistent Transactional States without a Global Clock, Avni & Shavit
% space suage: O(m) per thread.
% each transaction got a vector clock inddicating the ts at which other transactions started
% each location got a timestamp pair (x.ts,x.owner), where x.owner is the thread that wrote x last.
% upon reading x, T compare x.ts to  its ts and abort if x.ts[x.owner] > T.ts[x.owner]
% no forward reading possible, thus more false abprt:
% ``We argue that a TLC transaction will always fail if it attempts 
% to read a location that was written by some other transaction after it started.''
% example of spurious abort: w_p(x).c_p.r_q(x).w_q(y).c_p.r_{q'}(y).a_{q'}.r_{q'}(x).a_{q'}
