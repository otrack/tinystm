\section{A new consistency criterion}
\labsection{criterion}

This section is organized in four parts.
The first part (\refsection{criterion:model}) presents the elements of our system model.
Further, the notions of contention and binding (\refsection{criterion:bindings}) are introduced.
These notion are used to define stricter serializability (\refsection{criterion:spser}) and to study its applicability (\refsection{criterion:applicability}).

\subsection{System model}
\labsection{criterion:model}

Transactional memory (TM) is a recent paradigm that allows multiple processes to access concurrently a shared memory region.
Each process manipulates \emph{objects} in the shared memory with the help of transactions.
% object -> data item.

When a process starts a new transaction, it calls operation \stmBeginFunction.
Then, the process executes a sequence of \stmReadFunction and \stmWriteFunction operations on the shared objects according to some internal logic.
Operation \stmRead{x} takes as input an object $x$ and returns either a value in the domain of $x$ or a flag $\flagAbort$ to indicate that the transition aborts.
A write \stmWrite{x,v} changes $x$ to the value $v$ in the domain of $x$.
This operation does not return any value and it may also abort.
Without lack of generality, we consider that a transaction reads or writes to a given shared object at most once.
At the end of the transaction execution, the process calls \stmTryCommitFunction to complete the transaction.
This calls returns either $\flagCommit$, to indicate that the transaction was successful, or $\flagAbort$ if the transaction fails.
When the transaction aborts or commits, it takes no further step.

A \emph{history} is a sequence of invocations and responses of TM operations by one or more processes.
As illustrated with history $h_2$ below, a history is commonly depicted as parallel timelines, one per process, and each timeline consists of the sequence of transactions executed at that process.

\begin{figure}[!h]
  \centering
  \fontsize{8}{11}\selectfont
  \begin{tikzpicture}[scale=0.77]
    \node at (10,.3) {$(h_2)$};

    \node at (0.2,2.6) {$p$};
    \node at (0.2,1.8) {$q$};
    \node at (0.2,1) {$r$};

    \path[->] (0.5,2.6) edge (10,2.6);
    \path[->] (0.5,1.8) edge (10,1.8);
    \path[->] (0.5,1) edge (10,1);
    
    \path[callA] (2.2,2.6) edge (3.5,2.6);
    \path[callA,->] (2.2,2.8) edge (2.2,2.6);
    \path[callA,->] (3.5,2.6) edge (3.5,2.8);
    \node at (2.2,3) {$w_1(x_1)$}; 

    \path[callB] (1,1.8) edge (1.8,1.8);
    \path[callB,->] (1,2) edge (1,1.8);
    \path[callB,->] (1.8,1.8) edge (1.8,2);
    \node at (1,2.2) {$w_2(x_2)$}; 

    \path[callC] (4,1) edge (5,1);
    \path[callC,->] (4,1.2) edge (4,1);
    \path[callC,->] (5,1) edge (5,1.2);
    \node at (4,1.4) {$r_3(x_1)$};

    \path[callC] (6,1) edge (7,1);
    \path[callC,->] (6,1.2) edge (6,1);
    \path[callC,->] (7,1) edge (7,1.2);
    \node at (6,1.4) {$w_3(y_3)$};

    \path[callD] (3,1.8) edge (6.2,1.8);
    \path[callD,->] (3,2) edge (3,1.8);
    \path[callD,->] (6.2,1.8) edge (6.2,2);
    \node at (3,2.2) {$r_4(y_3)$}; 

    \path[callD] (8,1.8) edge (9,1.8);
    \path[callD,->] (8,2) edge (8,1.8);
    \node at (8,2.2) {$r_4(x_{?})$}; 
    
    \pgfresetboundingbox
    \clip[use as bounding box] (0,.7) rectangle (10,2.8);
  \end{tikzpicture}
\end{figure}


In history $h_2$, process $p$, $q$ and $r$ execute respectively transactions ${\color{blue}{T_1}}=w(x)$, ${\color{red}{T_2}}=w(x)$ then ${\color{Purple}{T_4}}=r(y);r(x)$, and ${\color{OliveGreen}{T_3}}=r(x);w(y)$.
In this history, transactions $T_1$, $T_2$ and $T_3$ all \emph{complete}, whereas transaction $T_4$ is \emph{live}.
For clarity, calls to $\stmBeginFunction$ are omitted in such diagrams and every complete transaction which is not explicitly aborted commits immediately after its last operation.

We note $\committed{h}$ the set of transactions that commit during history $h$.
In the case of history $h_2$, we have $\committed{h_2}=\{T_1,T_2,T_3\}$.

Over the course of its execution, a transaction $T_i$ reads and writes one or more versions of the shared objects.
The set of versions read by the transaction forms its \emph{read set} (denoted $\readSetOf{T_i}$).
The versions written during the execution define the \emph{write set} ($\writeSetOf{T_i}$).

A \emph{version} is the state of a shared object as produced by a write operation.
When a transaction $T_i$ writes to some object $x$, an operation denoted $w_i(x_i)$, it creates the version $x_i$ of $x$.
Versions allow to uniquely identify the state of an object as observed by a read operation, \emph{e.g.}, $r_3(x_1)$ in $h_2$.
We shall say that version $x_i$ is committed when $T_i$ commits in the history.

A history induces a real-time order between transactions (denoted $\hb_h$).
% TODO omit subscripts.
The order $T_i \hb_h T_j$ holds when $T_i$ completes in $h$ before $T_j$ begins.
For instance in history $h_2$, transaction $T_1$ precedes transaction $T_3$.
When transactions $T_i$ and $T_j$ are not related with real-time, they are \emph{concurrent} (denoted $T_i \cc_h T_j$).

History $h$ is \emph{complete} when it contains no live transaction.
Histories $h$ and $h'$ are \emph{equivalent} ($h \equiv h'$) when they contain the same events.
A history $h$ is \emph{sequential} when no two transactions are concurrent in $h$.
A sequential history $h$ is \emph{legal} when for every read $r_i(x_j)$ in $h$, $T_j$ is the last committed transaction before $T_i$ that writes to $x$ in $h$.

For some history $h$, a version order on object $x$ for $h$ is a total order over all the versions $x_i$ such that $T_i$ commits in $h$.
% TODO define everything without this (non-standard) assumption 
By extension, a version order for $h$ (written $\ll_h$) is a union of version orders, one per object.
As an example, in history $h_2$ above, we may consider the version order $(x_2 \ll_{h_2} x_1)$.

Consider a history $h$, some version order $\ll_h$ and two distinct transactions $T_i$ and $T_j$ in $h$.
Following \cite{Adya99}, we shall say that:
% FIXME in \cite{}, ``direct'' relations are only considered.
\begin{compactitem}
\item $T_j$ \emph{read-depend} on $T_i$ when $r_j(x_i)$ occurs in $h$ ($T_i \wrdep T_j$);
\item $T_j$ \emph{write-depends} on $T_i$ when $x_i \ll_h x_j$ is true ($T_i \wwdep T_j$); and
\item $T_j$ \emph{anti-depends} on $T_i$ when $r_i(x_k) \in h \land x_k \ll_h x_j$ holds ($T_i \adep T_j$).
\end{compactitem}
To illustrate the above definitions, consider again history $h_2$, with $x_2 \ll_h x_1$.
In this history, transaction $T_3$ read-depends on $T_1$ and $T_1$ write-depends on $T_2$.
Moreover, if transaction $T_4$ reads $x_2$, then $T_1$ anti-depends on $T_4$.

Consider that $T_j$ anti-depends on $T_i$ because it updates some object $x$ and $T_i$ reads a prior version.
This anti-dependency is a \emph{reverse-commit anti-dependency} (for short, RC-anti-dependency) \cite{hans16} when $T_j$ invokes $\stmTryCommitFunction$ before $T_i$ in $h$, and $T_i$ writes to some object $y \neq x$.
%
\footnote{
  By not requiring that $T_j$ reads $x$ before updating it, this definition slightly generalizes the notion of RC-anti-dependency as introduced in \cite{hans16}.
}

In what follows, we assume that every history is complete and that read operation always returns a committed version.
% r_i(x_j) \in h \implies \neg (r_i(x_j) \hb_h \stmTryCommitFunction_j)
These simplifying assumptions are motivated by the fact that TM designs favor wait-free operations and buffer updates until commit \cite{bookTM}.
% TODO remove these assumptions

\subsection{Contention and bindings}
\labsection{criterion:bindings}

Internally, a transactional memory is built upon a set of \emph{base objects}, such as locks and registers.
When two transactions are concurrent, their steps on these base objects interleave.
If two transactions access distinct objects and the TM is disjoint-access parallel, no contention occurs.
However, in the case they access the same base object, they may slow down each other.

A transactional read is \emph{invisible} when it does not change the state of the base objects implementing it.
With invisible reads, read contention is basically free.
From a performance point of view, this property is consequently appealing, since workloads exhibit a large ratio of read operations \cite{stampBenchmark}.

When two transactions are concurrently writing to the same object, it is possible to detect the contention and abort preemptively one of them.
On the other hand, when a read-write conflict happens, a \emph{race condition} occurs between the reader and the writer.
If the read operation takes place after the write, the reader is bound to use the version produced by the writer.
% when x_j is visible, all its dependencies are also visible.
% thus, if x_j is visible before T_i begins its execution, no dependency can be missed.
% this only happens when the two transactions are concurrent.

\begin{definition}[Binding]
  During a history $h$, when $T_i$ and $T_j$ are concurrent and $r_i(x_j)$ occurs in $h$, we say that $T_i$ is bound to $T_j$ on $x$.
\end{definition}

When transaction $T_i$ is bound to $T_j$, to preserve the consistency of its read set, $T_i$ must observe the dependencies of $T_j$.
This is for instance the case of transaction $T_4$ in history $h_2$, where this transaction is bound to $T_3$ on $y$.
As a consequence, $T_4$ must return $x_1$ as the result of its read on $x$, or its read set will be not be consistent.

\begin{definition}[Dependency]
  Consider some version order $\ll_h$ for $h$.
  We say that $T_j$ \emph{depends} on $T_i$ ($T_i \rdepends T_j$)  when $T_j$ read-depends or write-depends on $T_i$, directly or by transitivity.
  In other words, $\rdepends$ equals $\closureOf{((\wrdep) \union (\wwdep))}$, where $\closureOf{R}$ denotes the transitive closure of the binary relation $R$,
  For readability, we also define $T_j \depends T_i$ as $T_i \rdepends T_j$.
  % TODO to fix.
\end{definition}

Tracking dependencies is difficult for the TM since it requires to re-validate the read set after each read, rely on a global clock, or use large amount of metadata.
We observe that this computation is simplified when each version read before the binding is either a dependency of the version newly read, or was installed after it.
In which case, we will say that the binding is fair.

\begin{definition}[Fair binding]
  Consider that in some history $h$ a transaction $T_i$ is bound to a transaction $T_j$ on some object $x$.
  This binding is \emph{fair} when, for every version $y_k$ read by $T_i$ before $x_j$ in $h$, $T_j \depends_h T_k$ holds, or $T_k$ invokes $\stmTryCommitFunction$ after $T_j$.
\end{definition}

Going back to history $h_2$, the binding of $T_4$ to $T_3$ on $y$ is fair.
Indeed, this transaction did not read any data item before accessing the version of $y$ written by $T_3$.

When the binding is fair, the reader can leverage the metadata left by the writer to check prior versions it has read and ensure that its read set is consistent.
In the next section, we formalize this idea with the notion of stricter serializability.

\subsection{Stricter serializability}
\labsection{criterion:spser}

In what follows, we introduce and describe in detail \SPSER, the stricter serializability consistency criterion that we build upon in the remainder of this paper.
As strict serializability, \SPSER requires that committed transactions form a sequential history which preserves the real-time order.
In addition, it prohibits transactions to view inconsistencies unless one of their bindings is unfair.

For a history $h$ and some version order $\ll_h$, $<_h$ captures all the constrains on the ordering of transactions in $h$.
In other words, $<_h$ equals $((\wrdep) \union (\wwdep) \union (\adep) \union (\hb_h))$.
We note $<_{\committed{h}}$ the restriction of $<_{h}$ to the transactions committed in $h$.

\begin{definition}[Strict serializability \cite{Adya99}]
  The real-time serialization graph of history $h$ induced by $\ll_h$, written $\RSG(h,\ll_h)$, is defined as $(\committed{h},<_{\committed{h}})$.  
  History $h$ is \emph{strictly serializable} ($h \in \SSER)$ when $\RSG(h,\ll_h)$ is acyclic for a version order $\ll_h$.
\end{definition}

If history $h$ is strictly serializable, linearizing the transactions committed in $h$ following $<_{\committed{h}}$ leads to a legal history.
% TODO \cite{} ?
However, as pointed out in the introduction, \SSER does not constraint the behavior of aborted transactions which may behave arbitrary.
Opacity ($\OPA$) extends strict serializability to avoid such situations.

\begin{definition}[Opacity \cite{guerraoui2008correctness}]
  A history $h$ is opaque ($h \in \OPA$) when there exists a sequential and legal history $\lambda$ equivalent to $h$ that preserves the real-time order in $h$ (that is, $\hb_{\lambda} \subseteq \hb_{h}$).
\end{definition}

In this paper, we are interested in bridging the gap between strict serializability and opacity.
To this end, we need first to introduce the notion of strictly consistent snapshot. 

\begin{definition}[Strictly consistent snapshot]
  Consider some history $h$ and a version order $\ll_h$.
  A transaction $T_i$ observes a \emph{strictly consistent snapshot} for $\ll_h$ when the transitive closure of $<_h$ starting from $\{T_i\}$ is acyclic.
  % TODO for every read $r_i(x_{j \neq i})$ in $h$, $T_j$ commits in $h$, and
  %      writes also need to take place before, otherwise the prefix is invalid (theoretically this ofc is not a problem).
\end{definition}

Our \SPSER criterion requires that after each read operation, the transactions sees a strictly consistent snapshot or one of its bindings is unfair.

\begin{definition}[Stricter serializability]
  \labdef{spser}
  A history $h$ is stricter serializable ($h \in \SPSER$) when for some version order $\ll_h$,
  \begin{inparaenum}
  \item $\RSG(h,\ll_h)$ is acyclic, and
  \item for every transaction $T_i$ that aborts in $h$, either $T_i$ observes a strictly consistent snapshot for $\ll_h$, or one of its binding is unfair.
  \end{inparaenum}
\end{definition}

A strictly serializable history becomes opaque when aborted transactions read strictly consistent snapshots.
(We provide a rigorous proof for this claim in \refappendix{from}.)
As a consequence of the above definition, a stricter serializable history during which all the aborted transactions exhibit fair bindings is opaque.
The next section gives a class of applications for which this characterization is true.
In other words, we present a robustness criterion~\cite{CeroneG16} against $\SPSER$.

\subsection{Applicability}
\labsection{criterion:applicability}

In what follows, we detail the model of application we are interested with.
Then, we present our robustness criterion and prove that it applies to $\SPSER$.

\subsubsection{Model of application}
\labsection{criterion:app}

The state of an object commonly includes \emph{references} to one or more other objects in the shared memory.
These references between objects form the \emph{object graph} of an application.

When performing a computation, a process traverses the object graph.
To this end, the process knows initially an immutable root object in the graph.
Starting from this root, the process executes a traversal by using the references stored in each object.

For some transaction $T_i$, a \emph{path} is the sequence of versions $\pi_i$ read by $T_i$.
It should satisfy that 
\begin{inparaenum}
\item the first object in $\pi_i$ corresponds to the immutable root of the object graph, and
\item for all $x_j \in \pi_i$, some $y_k <_{\pi} x_j$ includes a reference to $x_j$.
\end{inparaenum}

A path $\pi_i$ is \emph{simple} when for all $k \in [1,\cardinalOf{\pi_i}]$, $\pi_i[k]$ includes a reference to $\pi_i[k+1]$.
When a path is simple, for some object $x$, $\pi_{i,x}$ and $\hat{\pi}_{i,x}$ are respectively the paths from the root to $x$ before and after transaction $T_i$.
Hereafter, when writing $\pi_{i,x}$, and $\hat{\pi}_{i,x}$, we implicitly assume that such a path exists.

\subsubsection{Robustness}
\labsection{criterion:robustness}

To define our robustness criterion, we focus specifically on $\SPSER$ implementations that allow invisible reads.
As pointed out earlier, this restriction is motivated by performance as most workloads are read-intensive.
In this context, the result of Hans et al. \cite{hans16} tells us that it is not possible to jointly achieve
\begin{inparaenum}
\item $\SSER$,
\item read invisibility,
\item minimal progressiveness, and
\item accept RC-anti-dependencies.
\end{inparaenum}
% TODO. this is an approximation of the real result of Hans et al. 
As a consequence, we remove histories that exhibit such a pattern from our analysis; hereafter, we shall note $\RCAD$ these histories.

Let us consider property $\mathcal{P}$ on a TM application as defined below.
Hereafter, we name $\mathcal{T}$ some set of transactions for which property $\mathcal{P}$ holds and let $H_{\mathcal{T}}$ refer to the histories built upon transactions in $\mathcal{T}$.
\begin{itemize}
\item[($\mathcal{P}$)]
  The object graph forms initially a tree and every transaction maintains this invariant.
  Moreover, every transaction in the application follows a simple path.
\end{itemize}
We wish to establish the following result:

\begin{proposition}
  \labprop{criterion:2}
  $(\SPSER \inter H_{\mathcal{T}} \setminus \RCAD) \subset \OPA$.
\end{proposition}

To state this result, we note $h$ some history  in $(\SPSER \inter H_{\mathcal{T}} \setminus \RCAD)$.
Since $h$ is serializable, there exists some sequential history $\lambda$ equivalent to $\committed{h}$.

\begin{lemma}
  \lablem{criterion:1}
  If $T_i \hb_{\lambda} T_j$ holds and $T_j$ reads $x$ in $\lambda$, then for every $y_k \in \hat{\pi}_{i,x}$, $T_j \depends_h T_k$ is true.
  % TODO. path wrt. \lambda
\end{lemma}

\begin{proof}

  Since $T_j$ reads $x$, $\pi_{j,x} \subseteq \readSetOf{T_j}$.
  Then, for some $y_k \in \hat{\pi}_{i,x}$, two cases are possible:
  \begin{itemize}
  \item ($\exists T_l \in \lambda : y_l  \in \pi_{j,x}$)
    The claim trivially holds if $l = k$.
    If now $l \neq k$, $T_i \hb_{\lambda} T_j$ implies that $T_k \hb_{\lambda} T_l$.
    Hence, $y_k \ll_{h} y_l$ is true leading to $T_j \depends_h T_k$.
  \item (Otherwise.)
    Let $T_l \hb_{\lambda} T_j$ be the transaction that deletes $y$ from the object graph.
    Without lack of generality assume this was still version $y_k$.
    Because $\pi_{j,x}$ is defined and property $\mathcal{P}$ holds, $\hat{\pi}_{l,x}$ is defined.
    By property $\mathcal{P}$, there exist $z_{?}$ and $t_{\versionAny}$ such that
    $\pi_{l,x} = \ldots z_{?} \ldots y_{k} \ldots t_{\versionAny} \ldots$
    and $\hat{\pi}_{l,x} = \ldots z_{l}t_{\versionAny} \ldots$, and transaction $T_l$ reads version $y_{k}$.
    Now, if $z_l$ is in $\pi_{i,x}$, $T_j \depends_h T_k$ holds.
    In the converse case, we repeat the above reasoning for $z_l$.
  \end{itemize}     
    
\end{proof}

\begin{corollary}
  \labcor{criterion:1}
  The relation $T_i \depends_h T_k$ holds for every version $y_k$ in ($\pi_{i,x} \union \hat{\pi}_{i,x}$).
\end{corollary}

\begin{lemma}
  \lablem{criterion:2}
  If transaction $T_i$ aborts in $h$ then all its bindings are fair.
\end{lemma}

\begin{proof}

  (By induction.)
  Define $x$ and $T_j$ such that $T_i$ is bound to $T_j$ on $x$ and assume that all the prior bindings (if any) of $T_i$ are fair.
  It follows that transaction $T_i$ observes a strictly consistent snapshot in $h$ before $r_i(x_j)$.
  Hence, there exists a committed transaction $T_k$ such that $\pi_{i,x}$ is the path to $x$ after transaction $T_k$ in $\lambda$ (i.e., $\hat{\pi}_{k,x}= \pi_{i,x}$).
  % FIXME $T_i$ observes a strictly consistent snapshot in $h'$ = h \setminus {the read operation after r_i(x_j) }.
  % FIXME prove this.
  Let us prove that $r_i(x_j)$ is fair.
  Depending on the relative positions of $T_j$ and $T_k$ in $\lambda$, there are two cases to consider.
  \begin{itemize}
  \item $(T_k \hb_{\lambda} T_j)$.
    This case is a direct application of \reflem{criterion:1}.
  \item $(T_j = T_k) \lor (T_j \hb_{\lambda} T_k)$.
    If $\pi_{i,x} = \hat{\pi}_{j,x}$ holds, then \refcor{criterion:1} tells us that $r_i(x_j)$ is fair.
    %
    Otherwise, there exists some transaction $T_l$ such that $T_j \hb_{\lambda} T_l \hb_{\lambda} T_k$ and $T_l$ modifies $\hat{\pi}_{j,x}$.
    Without lack of generality, assume $T_l$ to be the first such transaction.
    If $T_l$ invokes $\stmTryCommitFunction$ after transaction $T_j$, then the binding is fair.
    In the converse case, $T_l$ does not observe any modification made by $T_j$.
    % FIXME strictly consistent snapshot -> committed values only.
    Hence, $T_l$ modifies the path $\pi_{j,x}$.
    In order to maintain invariant $\mathcal{P}$, transaction $T_l$ must update some object $z$ read in $\pi_{j,x}$.
    Because $T_l$ invokes $\stmTryCommitFunction$ before $T_j$, yet $T_j \hb_{\lambda} T_l$, the two transactions are concurrent in $h$.
    Hence, history $h$ exhibits an RC-anti-dependency between $T_j$ and $T_l$ on object $z$.
    Contradiction.
  \end{itemize}
\end{proof}
