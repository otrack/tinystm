\section{A new consistency criteria}
\labsection{criteria}

In what follows, we present the elements of our system model, as commonly found in textbooks (e.g., \cite{}).
Then, we formulate our notion of stricter serializability.

\subsection{System Model}
\labsection{criteria:model}

Software transactional memory (STM) is a recent paradigm that allows multiple processes to access concurently a memory region.
Each process manipulates \emph{objects} in the shared memory with the help of transactions.
When a process begins a new transaction, it calls function \stmBeginFunction.
Then, it executes a sequence of \stmReadFunction and \stmWriteFunction operations on the shared objects according to some internal (not modeled) logic.
Operation \stmRead{x} takes as input an object $x$ and returns either a value in the domain of $x$, or a flag $\flagAbort$ to indicate that the transation aborts.
A write \stmRead{x,v} changes $x$ to the value $v$ in the domain of $x$.
This operation does not return any value and it may also abort.
At the end of the execution, the process calls \stmTryCommitFunction to terminate the transaction.
This calls returns either $\flagCommit$, to indicate that the transaction commits, or $\flagAbort$ if it aborts.

A \emph{history} is a sequence of invocations and responses of transactional operations by one or more processes.
As illustrated below, a history is commonly represented with parallel timelines, where each timeline represents the transactions executed by some processes.
In history $h_2$, process $p$, $q$ and $r$ execute respectively transactions ${\color{blue}{T_1}}=w(x)$, ${\color{red}{T_2}}=w(x)$ then ${\color{Purple}{T_4}}=r(y);r(x)$, and ${\color{OliveGreen}{T_3}}=r(x)$.
All the transactions but $T_4$ completes in this history.
For simplicity, complete transactions that are not explicitely aborted in a history commit immediately after their last operations.
We note $\committed{h}$ the set of transactions that commit during history $h$.
In the case of history $h_2$, we have $\committed{h_2}=\{T_1,T_2,T_3\}$.

\begin{figure}[!h]
  \centering
  \fontsize{8}{11}\selectfont
  \begin{tikzpicture}[scale=0.77]
    \node at (10,.3) {$(h_2)$};

    \node at (0.2,2.6) {$T_1$};
    \node at (0.2,1.8) {$T_2$};
    \node at (0.2,1) {$T_3$};

    \path[->] (0.5,2.6) edge (10,2.6);
    \path[->] (0.5,1.8) edge (10,1.8);
    \path[->] (0.5,1) edge (10,1);
    
    \path[callA] (2,2.6) edge (3,2.6);
    \path[callA,->] (2,2.8) edge (2,2.6);
    \path[callA,->] (3,2.6) edge (3,2.8);
    \node at (2,3) {$w_1(x_1)$}; 
    \node at (3,3) {};

    \path[callA] (4,2.6) edge (5,2.6);
    \path[callA,->] (4,2.8) edge (4,2.6);
    \path[callA,->] (5,2.6) edge (5,2.8);
    \node at (4,3) {$\flagCommit$};
    \node at (5,3) {};

    \path[callB] (1.5,1.8) edge (2.5,1.8);
    \path[callB,->] (1.5,2) edge (1.5,1.8);
    \path[callB,->] (2.5,1.8) edge (2.5,2);
    \node at (1.5,2.2) {$w_2(x_2)$}; 
    \node at (2.5,2.2) {};

    \path[callB] (3.5,1.8) edge (4.5,1.8);
    \path[callB,->] (3.5,2) edge (3.5,1.8);
    \path[callB,->] (4.5,1.8) edge (4.5,2);
    \node at (3.5,2.2) {$\flagCommit$};
    \node at (4.5,2.2) {};

    \path[callC] (5,1) edge (6,1);
    \path[callC,->] (5,1.2) edge (5,1);
    \path[callC,->] (6,1) edge (6,1.2);
    \node at (5,1.4) {$r_3(x_1)$};
    \node at (6,1.4) {};

    \path[callC] (7,1) edge (8,1);
    \path[callC,->] (7,1.2) edge (7,1);
    \path[callC,->] (8,1) edge (8,1.2);
    \node at (7,1.4) {$\flagCommit$};
    \node at (8,1.4) {};
    
    \pgfresetboundingbox
    \clip[use as bounding box] (0,.7) rectangle (10,2.8);
  \end{tikzpicture}
\end{figure}


A history induces a real-time order between transactions (denoted $\hb_h$).
This order holds between two transactions $T_i$ and $T_j$ when $T_i$ terminates in $h$ before $T_j$ begins.
For instance in history $h_2$, transaction $T_1$ precedes transaction $T_3$.
When two transactions are not related with real-time, they are \emph{concurrent}.

A \emph{version} is the state of a shared object as produced by the write of a transaction.
This means that when transaction $T_i$ writes to some object $x$, an operation denoted $w_i(x_i)$, it creates the version $x_i$ of $x$.
Versions allow to uniquely identify the state of an object as observed by a read operation, \emph{e.g.}, $r_3(x_1)$ in $h_2$.
When a transaction $T_i$ reads version $x_j$, we say that $T_i$ \emph{read-from} transaction $T_j$.

Given some history $h$ and some object $x$, a version order on $x$ for $h$ is a total order relation over the versions of $x$ in $h$.
By extension, a version order for $h$ is the union of all the version orders for all objects (denoted $\ll_h$).
For instance, in history $h_2$ above, we may consider the version order $(x_1 \ll_{h_2} x_2)$.

Consider an history $h$ and some version order $\ll_h$.
A transaction $T_i$ \emph{depends on} some transaction $T_j$, written $T_i \depends T_k$ when $T_j$ precedes $T_i$, $T_i$ reads-from $T_j$, or such a relation holds transitively.
Transaction $T_i$ \emph{anti-depends} from $T_j$ on object $x$, when $T_i$ reads a version $x_k$, $T_j$ writes object $x$, and $x_k$ precedes $x_j$ in the version order ($x_k \ll_{h} x_j$).
An anti-dependency between $T_i$ and $T_j$ on object $x$ is a \emph{reverse-commit anti-dependency} (for short, RC-anti-dependency) \cite{hans16} when $T_j$ commits before $T_i$ and $T_i$ writes some object $y \neq x$.

To illustrate the above definitions, consider again history $h_2$.
In this history, transaction $T_3$ depends on $T_1$ and $T_2$.
On the other hand, if $x_2 \ll_h x_1$ holds and $T_4$ reads $x_2$, then this transaction exhibits an anti-dependency with $T_1$.
This anti-dependency becomes an RC-anti-dependency if $T_4$ executes an additional step where it writes some object $z$.

A transaction observes a \emph{strictly consistent snapshot}~\cite{cc:coh:997} when it never misses the effects of some transaction it depends on.
In other words, the snapshot of transaction $T_i$ in history $h$ is strictly consistent when
\begin{inparaenum}
\item $T_i$ reads version $x_j$,
\item $T_k$ writes version $x_k$, and 
\item $T_i$ depends on $T_{k}$,
\end{inparaenum}
then version $x_k$ is followed by version $x_j$.
Formally, $((r_i(x_j) \in h) \land (w_k(x_k) \in h) \land (T_i \depends T_k)) \implies (x_k \ll_h x_j)$, where $\ll_{h}$ is some version order defined over history $h$.

\subsection{Contention and bindings}
\labsection{criteria:bindings}

Internally, a transaction memory is built upon a set of \emph{base objects}, such as locks or registers.
When two transactions are concurrent, their steps on these base objects are interleaving.
If the two transactions access disjoint objects and the TM is disjoint-access parallel, no contention occurs.
However, in the case they access the base object, they may slow down each other.

A transactional read is \emph{invisible} when it does not change the state of the base objects implementing it.
With invisible reads, read contention is basically free.
From a performance point of view, this property is consequently appealing, since workloads exhibit in most case a large ratio of read operations.

When two transactions are concurrently writing to some object, it is possible to detect the contention and abort preventively one of them.
On the other hand, when a read-write conflict occurs, a \emph{race condition} occurs between the reader and the writer.
If the read operation takes place after the write, the reader is bound to use the version produced by the writer.

\begin{definition}[Binding]
  When a transaction $T_i$ reads-from a concurrent transaction $T_j$ in a history $h$, we say that $T_i$ is bound to $T_j$ on $x$.
\end{definition}

When a transaction $T_i$ is bound to another transaction $T_j$, to preserve the consistency of its snapshot, $T_i$ must read the updates and causal dependencies of $T_j$ that are intersecting with its read set.
This is for instance the case of transaction $T_4$ in history $h_2$, where this transaction is bound to $T_3$ on $y$.
As a consequence, $T_4$ must return $x_1$ as the result of its read on $x$, or its snapshot will be inconsistent.

Tracking this causality relation is difficult for the conention manager as it requires to inspect the readset, rely on a global clock, or use large amount of metadata \cite{}.
We observe that it is simpler if the items read prior the binding are either, also accessed by the writer, or by one of its dependencies.
In which case, we will say that the binding is fair.

\begin{definition}[Fair binding]
  Consider that in some history $h$ a transaction $T_i$ is bound to a transaction $T_j$ on some object $x$.
  This binding is \emph{fair} when every object $y$ read (but not written) by $T_i$ before $x$ in $h$ is also accessed by $T_j$.
\end{definition}

Going back to history $h_3$, the binding of $T_4$ to $T_3$ on $y$ is fair.
Indeed, this transaction did not read any data item before accesing the version of $y$ written by $T_4$.
When the binding is fair, the reader can leverage the metadata left by the writter to check prior versions it has read and ensure the consistency of later read operations.
In the next section, we formalize this idea with the notion of stricter serializability.

\subsection{Stricter serializability}
\labsection{criteria:spser}
Here we introduce and describe in details \SPSER, the stricter serializability consistency criteria that we build upon in the remainder of this paper.
The \sser criteria requires that committed transactions form a sequential history.
In addition, \sser prohibits transactions to view inconsistencies unless one of their bindings is unfair.
Similar to related work~\cite{syn:db:1467}, we give a graph characterization of stricter serializability.

\begin{definition}[Strict serialization graph]
  Consider some version order $\ll_h$ for $h$.
  As defined below, relation $<$ captures all the relations over $\committed{h}$ induced by $\ll_h$.
  The serialization graph of history $h$ induced by $\ll_h$, written $\SSG(h,\ll_h)$, is $(\committed{h},<)$.
  \begin{displaymath}
    \begin{array}{l}
      T_i < T_{j \neq i}  \equaldef \\
      \lor~ T_i \hb_h T_j {\hspace{16.8em}\text{(1)}} \\
      \lor~ \exists x : \lor~ r_j(x_i) \in h {\hspace{13.3em}\text{(2)}} \\
      \hspace{2.9em}\lor~ \exists T_k : x_k \ll_h x_j \land \lor~T_k = T_i {\hspace{5em}\text{(3)}} \\
      \hspace{12.2em} \lor~ r_i(x_k) \in h {\hspace{3.9em}\text{(4)}}
    \end{array}
  \end{displaymath}  
\end{definition}
In the above definition, (1) is a real-time order between $T$ and $T'$, (2) a read-write dependency, (3) a version ordering, and (4) an anti-dependency.

\begin{definition}[Stricter serializability]
  A history $h$ is stricter serializable ($h \in \SPSER$) when
  \begin{inparaenum}
  \item for some version order $\ll_h$, the serialization graph $(\committed{h},<)$ is acyclic, and
  \item for every transaction $T_i$ that aborts in $h$, either $T_i$ observes a strictly consistent snapshot in $h$, or one of its bindings is unfair.
  \end{inparaenum}
\end{definition}

Opacity ($\OPA$) and strict serializability ($\SSER$) coincide when no transaction abort during an history.
As a consequence of the above definition, if in a stricter serializable history all the aborted transaction exhibit fair bindings, the history is opaque.

\begin{corollary}
  \labcor{criteria:1}
  If ($h \in \SPSER$) exhibits no unfair bindings then ($h \in \OPA$) holds.
\end{corollary}

\subsection{Applicability}
\labsection{criteria:applicability}

\refcor{criteria:1} offers a convenient property on histories that, when it aplies, allows to reach opacity.
Based on this property, we detail in this section a workload for which this property applies.
In other words, we give a robustness criteria \cite{alexey} against $\SPSER$.

To conduct our analysis, we focus specifically on $\SPSER$ implementations that allow invisible reads.
As pointed out at the begining of this section, this restriction is motivated by perfomance since most workloads are read-intensive.
In this context, the result of \citet{hans16} tells us that it is not possible to jointly achieve
\begin{inparaenum}
\item $\SSER$,
\item read invisibility,
\item minimal progressiveness, and
\item accept RC-anti-dependencies.
\end{inparaenum}
As a consequence, we have to prune histories that exhibits such patterns from our analysis;
hereafter, we shall note these histories $\RCAD$.

\subsubsection{A robustness criteria}
\labsection{criteria:robustness}

The state of an object commonly include a reference to another object.
These references between objects form the \emph{object graph} of the application.
Processes know initially one (or more) \emph{root} objects in this graph, and upon performing a computation traverse the object graph appropriately.

Consider that a transaction $T$ access in order objects $x_1, \ldots, x_m$.
We say that $T$ cross a \emph{simple path} if for all $i \in [1,m-1]$, the state of $x_i$ includes a reference to $x_{i+1}$.
At the light of this definition, we call $P$ the following property on a workload.

\begin{itemize}
\item[($P$)]
  The object graph forms initially a forest and every transaction maintains this invariant.
  Moreover, every transaction in the workload crosses a simple path.
\end{itemize}

The proposition that follows proves that, provided $\mathcal{P}$ holds and the TM does not accept RC-anti-dependencies, then the workload is robust against $\SPSER$.

\begin{proposition}
  Let $\mathcal{T}$ be some set of transactions for which property $P$ holds.
  Define $H_{\mathcal{T}}$ the histories built upon transactions in $\mathcal{T}$.
  It is true that: $(\SPSER \inter H_{\mathcal{T}} \setminus \RCAD) \subset \OPA$.
\end{proposition}

\begin{proof}
  Choose some history $h \in H_{\mathcal{T}} \inter \SPSER$, and let $T_i$ be a transaction that aborts in $h$.
  In what follows, we prove all of the bindings of $T_i$ are fair in $h$.
  By definition of $\SPSER$, this leads to the fact that $h \in \OPA$.

  We proceed by induction.
  Define $x$ and $T_j$ such that $T_i$ is bound to $T_j$ on $x$ and assume that all the prior bindings of $T_i$ are fair.
  Let $\pi_i$ be the path taken by $T_i$ to read $x$.
  Similarly, define $\pi_j$ as the path taken by $T_j$ before object $x$.
  By hypothesis, all the bindings before $x$ are fair for $T_i$.
  Since $h \in \SPSER$, the snapshot read by $T_i$ is strictly consistent up to $x$.

  Let us note $\lambda$ the linerization of the transactions that commit in $h$.
  We know that history $\lambda$ is of the form $\lambda=T_{a_1} \ldots T_{a_j} T_j T_{a_{j+}} \ldots T_{a_m}$.
  Let us observe that $\pi_j$ equals the path to object $x$ after transaction $T_{a_j}$.
  Similarly for some $T_{a_i}$, $\pi_i$ is the path to $x$ after transaction $T_{a_i}$.  
  According to the positions of $T_{a_i}$ and $T_{a_j}$ in $\lambda$, we may consider the following cases:
  \begin{itemize}
  \item ($T_{a_j} = T_{a_i}$)
    In this case, the two paths are equals.
    Consequently all the object reads by $T_i$ are also accessed by $T_j$.
    It follows that the binding is fair.    
  \item ($T_{a_j} <_{\lambda} T_{a_i}$)
    Name $T$ the first transaction that modifies the path $\pi_j$.
    Since $T_i$ observes a strictly consistent snapshot up to $x$ in $h$, $T$ commits before $T_i$ in $h$.
    Now, observe that $T_j$ is concurrent to $T_j$, thus it is also concurrent to $T$.
    Both $T$ and $T_j$ starts from the same root object, $T$ modifies the path $\pi_j$ and commits before $T_j$,
    Since $T$ writes $x$, history $h$ exhibits and RC-anti-dependency between $T$ and $T_j$.
    Contradiction.
  \item ($T_{a_i} <_{\lambda} T_{a_j}$)
    Let $y$ be an object read by $T_i$ in the path $\pi_i$ and assume that $y$ is no more in $\pi_j$.
    Name $T \leq_{\lambda} T_{a_j}$ the transaction that removes $y$ from the path leading to $x$.
    To unlink $y$, while maintaining the forest property of the object graph, $T$ must make a bridge between an object $y'$ before $y$ and an object $y''$ after it.
    As a consequence, this transaction reads $y$ when executing this operation.
    Then, because $T$ appears before $T_j$ in $\lambda$ and $h \notin \RCAD$, necessarily $T$ is a dependency of transaction $T_j$.
  \end{itemize}
\end{proof}
